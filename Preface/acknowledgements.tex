\chapter*{Acknowledgements}

\addcontentsline{toc}{chapter}{Acknowledgements}

I would like to start by acknowledging Dr. Jonathan Spanier for all of the all of the guidance that he has given to me over the course of my tenure at Drexel University. Allowing me to enter into the research side of school as a young freshman with stars in his eyes was undeniably one of the defining points in my academic career. The variety of skills I have been able to learn while working with him, and of course all of the members of his extremely talented group, are sure to prove to be invaluable during the next stages of life after Drexel. While I learned plenty of facts and received valuable training in an incredible range of technical equipment, what really stands out is the manner in which I was taught how to think. How to think about puzzling results that didn't quite add up evenly. How best to plan experiments to maximize the data obtained from it. Analyzing the remnants from when things went wrong, and figuring out how to prevent such things from happening in the future. The researcher's mindset is one of inquisitiveness and analysis and a drive to truly understand, and it has been of great value both in and out of the laboratory. 

Another pair of honorable mentions goes to Dr. Stephen Nonnenmann and Dr. Eric Gallo. These two served me as would-be mentors as I tried to get my feet under me in my first year or so. Always around to bounce questions off of, quick with helpful advice, and utterly filled to the brim with useful knowledge. Between these two, there was nothing that couldn't get done or be answered in the lab. These two set me up with the knowledge and mental tools I would need to go off exploring on my own journeys into the wild seas of science. I cannot thank them enough. 

Then there was the rest of the stalwart crew. Stephanie Johnson, Oren Leaffer, Terrance McGuckin, Guannan Chen, and Christopher Hawley. And I couldn't forget my fellow undergrads Dominic Bruzzese and Michael Coster. The great conversations --- especially with Oren (whom I can still count on for some useful advice about any question) even if they tended to get a little long-winded at times --- that I had with these characters will be some of my best memories from my time here. Each of us had our own idiosyncrasies, but I like to think we made a fantastic team when you put us all together. 

Of course, I would also like to send a heartfelt thank you to three more individuals who worked closely with my project at three very different stages. Rahul Joseph was my mentor when I first joined the lab, and it was a derivative of his project that over time metamorphosed into the thesis sitting before you.  Dr. Greg Soja was a post-doctorate fellow who worked alongside me on this project as well, and helped solve some rather difficult problems that we encountered. Last, but far from least, is Dr. Maria Sancho-Torres. She brought a bit of spark and a new point of view to the project that, again, helped sort out some problems that otherwise might have stymied me for a lot longer than they should have. 

I must also thank Keith Fahnstock and Dr. Caroline Schauer for allowing me and helping me with the use of their ellipsometer. Keith in particular, for his training and just good conversation (both on and off topic) when data collection got boring. In the same vein, I must gratefully acknowledge the assistance that Dr. Amy Fahnstock and her advisor Dr. Giuseppe Palemese provided me by allowing access to their TGA and DSC instrumentation that was pivotal to the success of this project. 

Of course, one does not live in a bubble, and I must thank the rest of the excellent faculty of the Materials Engineering Dept. for all of their efforts (both for their informative courses, but primarily for the discussions we had outside the classroom). I feel that I need to particularly acknowledge Dr. Steve May and Dr. Antonios Zavaliangos in this regard. 

Along the lines of the opportunity I have had during this past year abroad, I would like to personally thank Dr. Carlo Casari, Politecnico di Milano, and everyone who had a hand in making the E.A.G.L.E.S. program come together. And of course the scores of crazy friends that I picked up along the way. 

This has been a lot of talk already, but I personally feel I saved the most important for last.  My father, Bruce, and my mother, Amy, deserve more thanks than I think they realize. For being the supports that I could fall back on, for sharing in both the little successes and the big triumphs, for always letting me talk about what I was working on (and oh the respect I have for trying to learn about it from me...), and for always pushing me to go beyond what I thought I was capable of. Mom and Dad, thank you so very much. This one's for you guys. Now on with the show. 











