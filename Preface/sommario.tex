\newpage
\chapter*{Sommario}

\addcontentsline{toc}{chapter}{Sommario}

%%%%%%%%%%%%%%%%%%%%%%%%%

In this study, a method is proposed for the development of a novel ALD thin-film deposition process. This involves the identification, testing, and selection of precursors; analysis of growth parameters and their effects on the produced films; and in-depth \emph{ex situ} characterization of prepared films via a variety of techniques. 

The material under consideration is lead titanate (PTO, \PTO{}), which is an end group of the lead zirconate titanate family (PZT, \PZT{}). Lead titanate is a complex perovskite oxide, which exhibits ferroelectricity at room temperature and has a high Curie temperature at which it transitions to paraelectric behavior. Ferroelectricity is an interesting material property, as it involves a hysteretic behavior with regard to the polarization of the material. This allows the material to switch from one state to another and recall this stored polarization even without the presence of an electric field. This property has a wide variety of applications in electronics and other physical sciences. 

There are a large number of potential deposition techniques for thin film structures. Some, such as sol-gel processing, work on the principles of wet chemistry and allow for the formation of films with a tightly controlled composition. However, such techniques come with all of the drawbacks of solution based methods. Chemical vapor deposition (CVD) is another very common deposition method for thin films. It has a number of advantages, such as the capability to epitaxially deposit films (which greatly improves interface quality). However, it is not easy to precisely control film thicknesses and conformally coating non-planar structures is a very difficult task. 

Atomic layer deposition (ALD) is a modification on standard CVD processes. It separates the reaction between the metallic precursor and the oxidant into two half reactions. These half reactions are only capable of occurring on the surface of the substrate, as opposed to occurring in the gas phase. By separating the reactions a number of important changes happen. One of the most critical, and a hallmark of an ALD deposition, is that the reaction becomes self-limiting. A half reaction can only deposit at most a single monolayer of material. This gives the incredibly high thickness resolution that can only be obtained by a very few processes. ALD is capable of resolutions down to the angstrom level (1 \AA\ = 0.1 nm). This self-limiting behavior also gives ALD some of its other advantages, such as the ability to precisely and conformally coat a complex three-dimensional structure with the same high resolution. This is nearly impossible with any other deposition process. 

ALD precursors are more particular than those for CVD. They must be reactive with the surface and with an oxidizer, but they cannot be capable of reacting with themselves. The temperature required to activate the reaction with the surface must be lower than the temperature required to pyrolytically crack the molecule. Finally, at the end of each pair of half reactions new surface sites capable of reacting with the precursor must be repopulated with the proper ligands. 

A commonly used ALD precursor for titanium is readily available: titanium(IV) isopropoxide. However, there is no single ideal ALD precursor for lead. Of a number of different compounds two were chosen for further inspection: bis(2,2,6,6-tetramethyl-3,5-heptanedionato) lead(II) (\TMHD{}) and lead(II) hexafluoroacetylacetonate (\HFAc{}). 

The two selected precursors were analyzed using thermogravimetric analysis (TGA) as well as differential scanning calorimetry (DSC). Between the two techniques it was possible to identify the regions of ideal volatility as well as a potentially viable temperature range where ALD reactions could take place. From this analysis, it was expected that \TMHD{} would perform better in an ALD process than \HFAc{}. 

A number of samples were prepared and analyzed. Variable angle spectroscopic ellipsometry was used to measure film thicknesses and growth rates, along with some optical and electronic properties. Calibrated X-ray fluorescence spectroscopy (XRF) was used to obtain quantitative compositional analyses on the samples. X-ray diffraction (XRD) was used to investigate the phases present and their relative concentration in the final samples. 

In concluding this study, it was found that it is possible to grow complex perovskite oxides via ALD. However obtaining crystallized samples of pure perovskite phase was non-trivial and requires more investigation to fully accomplish. However, the study did manage to develop a valuable procedure for developing ALD processes of various materials. 

Further work would include analysis of the ferroelectric behavior of the films, improving phase purity in the final samples, and extension of the method to creating depositions for other material systems such as \PZT{} or barium strontium titanate (\ce{BaSrTiO3}). 







