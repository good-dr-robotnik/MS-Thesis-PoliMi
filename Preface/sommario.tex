\newpage
\chapter*{Sommario}

\addcontentsline{toc}{chapter}{Sommario}

In questo studio, viene proposto un metodo per lo sviluppo di un nuovo processo di deposizione Atomic layer depositon (ALD) per film sottili. Questo include l�identificazione, la verifica e la selezione dei precursori; l�analisi dei parametri di crescita e dei loro effetti sui film prodotti e la caratterizzazione ex situ mediante varie tecniche.

	Il materiale in esame \`e titanato di piombo (PTO, \PTO{}), della famiglia del titanato zirconato di piombo (PZT, \PZT{}). Il titanato di piombo \`e un ossido perovskitico complesso, che presenta ferroelettricit\`a a temperatura ambiente ed ha una elevata temperatura di Curie alla quale transisce  al comportamento paraelettrico. La ferroelettricit\`a \`e una propriet\`a interessante, in quanto comporta un comportamento isteretico della polarizzazione del materiale. Questo permette al materiale di passare da uno stato all�altro e di richiamare la polarizzazione memorizzato anche senza la presenza di un campo elettrico. Questa propriet\`a \`e di interesse per una vasta gamma di applicazioni nel campo dell�elettronica e delle altre scienze fisiche. 

	Ci sono un gran numero di tecniche di deposizione per strutture a film sottile. Alcune, quali sol-gel, si basano su principi di �wet chemistry�  e consentono la formazione di film con una composizione altamente controllata. Tuttavia, tali tecniche possiedono tutti gli inconvenienti dei metodi chimici a base di soluzioni. La deposizione chimica da fase vapore (CVD) \`e un altro metodo molto comune di deposizione di film sottili. Esso ha una serie di vantaggi, come ad esempio la capacit\`a di depositare film epitassiali  (con un notevole miglioramento della qualit\`a delle interfacce). Tuttavia non \`e facile controllare con precisione lo spessore e rivestire in maniera conforme strutture non planari \`e un compito molto difficile. La ALD \`e una tecnica basata su una modifica del processo CVD. Si separa una reazione tra il precursore metallico e l�ossidante in due semireazioni. Queste semireazioni sono in grado di verificarsi solo sulla superficie del substrato e non in fase gas. La separazione  delle reazioni porta ad una serie di importanti cambiamenti. Una delle caratteristica pi\`u critiche, della deposizione ALD, \`e che la reazione diventa auto-limitante. Una semireazione puo� depositare al massimo un monostrato di materiale. Questo permette una risoluzione nel controllo dello spessore che pochissime altre tecniche possiedono. La ALD \`e in grado di controllare lo spessore fino al livello dell�angstrom (1 \AA\ = 0.1 nm). Questo comportamento autolimitante d\`a anche alla ALD alcuni altri vantaggi, quali la possibilit\`a di rivestire con precisione e in modo conforme una struttura complessa tridimensionale mantenendo la stessa risoluzione. Questo \`e quasi impossibile con qualsiasi altro processo di deposizione. I precursori usati nella ALD sono pi\`u particolare di quelli tipici per CVD. Essi devono essere reattivi con la superficie e con un ossidante, ma non devono essere in grado di reagire tra dio loro. La temperatura necessaria per attivare la reazione con la superficie deve essere inferiore alla temperatura di pirolisi necessaria per rompere la molecola. Infine, al termine di ciascuna coppia di semireazioni  \`e necessario ripopolare i siti superficiali con i ligandi appropriati capaci di reagire con il precursore. Un precursore per ALD comunemente usato per il titanio \`e il titanio(IV) isopropossido, mentre  se non esiste un precursore ALD ideale per il piombo. Di una serie di due diversi composti sono stati scelti per ulteriori indagini il bis (2,2,6,6-tetrametil-3 ,5-heptanedionato) piombo (II) (\TMHD{}) e piombo(II) hexafluoroacetylacetonate (\HFAc{}). I due precursori selezionati sono stati analizzati utilizzando analisi termo gravimetrica (TGA) e calorimetria a scansione differenziale (DSC). Tra le due tecniche \`e stato possibile identificare le regioni di volatilit\`a ideale e un intervallo di temperatura in cui potrebbe avvenire le reazioni potenzialmente vitali per il processo ALD. Da questa analisi, ci si aspettava che \TMHD{} si comporta meglio di \HFAc{} in un processo di ALD. 

	Alcuni campioni sono stati preparati ed analizzati. La ellissometria spettroscopica ad angolo variabile\`e stata utilizzata per misurare gli spessori dei film e i tassi di crescita, insieme ad alcune propriet\`a ottiche ed elettroniche. La spettroscopia di fluorescenza a raggi X (XRF) \`e stata utilizzata per ottenere analisi compositive quantitative sui campioni. La diffrazione di raggi X (XRD) \`e stata utilizzata per studiare le fasi presenti e la loro relativa concentrazione nei campioni finali. A conclusione di questo studio, si \`e scoperto che \`e possibile far crescere ossidi di perovskite complessi tramite ALD, ma di ottenere campioni cristallizzati di pura fase perovskite  non \`e banale e richiede ulteriori studi. Tuttavia, durante questo lavoro di tesi si \`e riuscito a sviluppare una procedura importante per lo sviluppo di processi ALD di vari materiali. Ulteriori lavori comprendono l�analisi del comportamento ferroelettrico dei film, migliorare la purezza di fase nei campioni finali, e l�estensione del metodo di deposizione per creare sistemi di altro materiale come \PZT{} o titanato di stronzio e bario ((Ba,Sr)Ti\ce{O3}).
