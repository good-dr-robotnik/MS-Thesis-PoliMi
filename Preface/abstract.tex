\newpage
\chapter*{Abstract}
\addcontentsline{toc}{chapter}{Abstract}

%%%%%%%%%%%%%%%%%%%%%%%%%%%%%%%%%%%%%%%%%%%%%%%%%%%%
%%%%%%%%%%%%%%%%%%%%%%%%%%%%%%%%%%%%%%%%%%%%%%%%%%%%
%%%%%%%%%%%%%%%%%%%%%%%%%%%%%%%%%%%%%%%%%%%%%%%%%%%%

\noindent As devices based on ferroelectric films become more commonplace, a commercially viable process for fabricating the material is needed; low cost and high volume are implied necessities for such a process. This work focuses on the application of atomic layer deposition (ALD) to this task. ALD is a standard fabrication process in the electronics industry, valuable for its high uniformity across large surfaces, capability to control film thickness with very high resolution, and conformality across three dimensional structures. The various tasks in designing and optimizing an ALD process, as well as characterization methods for analysis of the produced films, are discussed here in detail.

The primary focus of this work is the development of a process for depositing lead titanate (\ce{PbTiO3}) thin films via an ALD process. Research focused on the application of bis(2,2,6,6-tetramethyl-3,5-heptanedionato)lead(II) (Pb(TMHD)$_{2}$) toward this end. 

Work supported by the Army Research Office under Grant \#W911NF-08-1-0067.