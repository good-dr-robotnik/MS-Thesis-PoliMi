\mbox{}\thispagestyle{empty}
\cleardoublepage
\chapter*{Abstract}
\addcontentsline{toc}{chapter}{Abstract}

%%%%%%%%%%%%%%%%%%%%%%%%%%%%%%%%%%%%%%%%%%%%%%%%%%%%
%%%%%%%%%%%%%%%%%%%%%%%%%%%%%%%%%%%%%%%%%%%%%%%%%%%%
%%%%%%%%%%%%%%%%%%%%%%%%%%%%%%%%%%%%%%%%%%%%%%%%%%%%

Advanced technologies derive many of their capabilities from the advanced materials that they are made from. Complex oxides are a class of materials which are driving technological advancement in a host of different directions. These highly functional materials have a great variety of useful properties, which can be chosen and even engineered through careful choice of material. 

Advanced materials require advanced deposition methods. Atomic layer deposition (ALD), a variant of chemical vapor deposition (CVD), is gaining more use in industry for its ability to provide ultra-high film thickness resolution (down to 0.1 nm), capability to conformally coat three-dimensional structures, and its high uniformity across large surface areas. Additionally, ALD processes provide a possibility to improve economic and environmental viability of the process as compared to CVD by using and wasting less toxic reactants and expelling fewer nano-particulate byproducts. 

ALD processes are highly mature for many binary oxides commonly used in the semiconductor industries, however processes for depositing heavy metal oxides and complex oxides --- oxides containing two or more separate metallic cations --- are sorely lacking in literature. 

The primary focus of this work is the development of a process for depositing the complex perovskite oxide lead titanate (\PTO{}), an end group of the lead zirconate titanate family (\ce{PbZr_{x}Ti_{1-x}O3}), which has valuable technical applications as well as serves as a template for applying this research into other material systems. 

The author gratefully acknowledges the Army Research Office (ARO) for their support of this project under the funding provided by Grant \# W911NF-08-1-0067.


%
%\reword{XXXXXXXXXXXXXXXX}
%
%As devices based on ferroelectric films become more commonplace, a commercially viable process for fabricating the material is needed; low cost and high volume are implied necessities for such a process. This work focuses on the application of atomic layer deposition (ALD) to this task. ALD is a standard fabrication process in the electronics industry, valuable for its high uniformity across large surfaces, capability to control film thickness with very high resolution, and conformality across three dimensional structures. The various tasks in designing and optimizing an ALD process, as well as characterization methods for analysis of the produced films, are discussed here in detail.
%
%The primary focus of this work is the development of a process for depositing lead titanate (\ce{PbTiO3}) thin films via an ALD process. Research focused on the application of bis(2,2,6,6-tetramethyl-3,5-heptanedionato)lead(II) (Pb(TMHD)$_{2}$) toward this end. 
%
%Work done towards the completion of this study was supported by the Army Research Office under Grant \#W911NF-08-1-0067. 