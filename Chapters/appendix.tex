\chapter{Supplemental Information}
\label{chap:appendix}
\thispagestyle{empty}

%%%%%%%%%%%%%%%%%%%%%%%%%%%%%%%%%%%%%%%%%%%%%%%%%%%%
%%%%%%%%%%%%%%%%%%%%%%%%%%%%%%%%%%%%%%%%%%%%%%%%%%%%
%%%%%%%%%%%%%%%%%%%%%%%%%%%%%%%%%%%%%%%%%%%%%%%%%%%%

\section{List of Chemicals}
\label{sup:LoChemicals}

\ce{Ph4Pb}

\ce{Ti-o-i-Pr}

\ce{Pb(TMHD)2}

\ce{Pb(HFAc)2} %Not correct name

Buffered \ce{HF}

\ce{N2}

\ce{O2}

\ce{O3}

\ce{H2O}

IPA

Acetone

%%%%%%%%%%%%%%%%%%%%%%%%%%%%%%%%%%%%%%%%%%%%%%%%%%%%
%%%%%%%%%%%%%%%%%%%%%%%%%%%%%%%%%%%%%%%%%%%%%%%%%%%%
%%%%%%%%%%%%%%%%%%%%%%%%%%%%%%%%%%%%%%%%%%%%%%%%%%%%

\begin{landscape}
\section{List of Samples}
\label{sup:LoSamples}


%\begin{table}[htdp]\tiny
%\begin{center}
%\begin{longtable}{|>{\centering}m{1.6cm}|>{\centering}m{1.5cm}|>{\centering}m{1.2cm}|>{\centering}%
%	m{1.2cm}|>{\centering}m{0.9cm}|>{\centering}m{1.0cm}|>{\centering}m{0.9cm}|>{\centering}%
%	m{0.9cm}|>{\centering}m{0.9cm}|>{\centering}m{0.7cm}|>{\centering}m{6cm}|}
%	%
%	\hline
%	\textbf{Temperature (\degC{})} & \textbf{Pb:Ti  Deposition Ratio} & \textbf{Run Number} &%
%	\textbf{Substrate Type} & \textbf{Super-Cycles} &\textbf{Total Cycles} & \textbf{Anneal Type} &%
%	\textbf{Anneal Temp (\degC{})} & \textbf{Anneal Time} & \textbf{XRD} & \textbf{Notes}%
%	\endhead
%	\hline\hline
%	0&a&b&c&d&e&f&g&h&i&\\ \hline
%	&&&&&&&&&&\\ \hline
%	&&&&&&&&&&\\ \hline
%	&&&&&&&&&&\\ \hline
%	&&&&&&&&&&\\ \hline
%	&&&&&&&&&&\\ \hline
%	&&&&&&&&&&\\ \hline
%	&&&&&&&&&&\\ \hline
%	%
%\end{longtable}
%\caption[List of Samples]{A list of samples used during the course of this work}
%\end{center}
%\label{default}
%\end{table}%

{\tiny
\begin{longtable}{|>{\centering}m{1.6cm}|>{\centering}m{1.5cm}|>{\centering}m{1.2cm}|>{\centering}m{1.2cm}|>{\centering}m{0.9cm}|>{\centering}m{1.0cm}|>{\centering}m{0.9cm}|>{\centering}m{0.9cm}|>{\centering}m{0.9cm}|>{\centering}m{0.7cm}|m{6cm}|}
	\caption[List of Samples]{A list of samples used during the course of this project. \label{tbl:LoSamples}}\\
	\hline
	\textbf{Temperature (\degC{})} & \textbf{Pb:Ti  Deposition Ratio} & \textbf{Run Number} & \textbf{Substrate Type} & \textbf{Super-Cycles} & \textbf{Total Cycles} & \textbf{Anneal Type} & \textbf{Anneal Temp (\degC{})} & \textbf{Anneal Time} & \textbf{XRD} & \textbf{Notes}\\ \hline \hline \endfirsthead%
	\caption[]{A list of samples used during the course of this project. \label{tbl:LoSamples}}\\
	\hline
	\textbf{Temperature (\degC{})} & \textbf{Pb:Ti  Deposition Ratio} & \textbf{Run Number} & \textbf{Substrate Type} & \textbf{Super-Cycles} & \textbf{Total Cycles} & \textbf{Anneal Type} & \textbf{Anneal Temp (\degC{})} & \textbf{Anneal Time} & \textbf{XRD} & \textbf{Notes}\\ \hline \hline \endhead%
	200 & 1:1 & 3 & Si & 250 & 500 &  &  &  & N & Blah  \\ \hline
	200 & 1:2 & 2 & Si & 250 & 750 &  &  &  & N &  \\ \hline
	\multirow{2}{*}{200} & \multirow{2}{*}{3:1} & \multirow{2}{*}{30} & Si & \multirow{2}{*}{160} & \multirow{2}{*}{640} &  &  &  & N & \multirow{2}{*}{} \\  \cdashline{4- 4}\cdashline{7- 10}
	& & & Pt & &  &  &  &  & N &  \\ \hline 
	\multirow{3}{*}{250} & \multirow{3}{*}{1:1} & \multirow{3}{*}{0} & Si & \multirow{3}{*}{625} & \multirow{3}{*}{1250} & Oven -- Amb & 650 & 2h & Y & Green sample, no apparent change \\* \cdashline{4- 4}\cdashline{7- 11}
	  &   &   & Si &   &   & Oven -- Amb & 900 & 2h & Y & Sample turned a purplish blue color, still shiny but not as shiny as untreated \\* \cdashline{4- 4}\cdashline{7- 11}
	  &   &   & Si &   &   & RTA & 900 & 10m & N & Small chip, no longer shiny, odd structures present after annealing (via opt. micro), appears to have formed nano-rods after anneal (see SEM images).  Made more samples in RTA at 600, 700, 800, 900\degC{}, similar results on all but 600\degC{} test \\ \hline 
	250 & 1:1 & 1 & Si & 475 & 950 &  &  &  & N &  \\ \hline 
	250 & 1:2 & 6 & Si & 250 & 750 &  &  &  & N &  \\ \hline 
	250 & 3:1 & 13 & Si & 250 & 1000 &  &  &  & N & Tried 60s between Pb pulses, needs a little more. Sample appeared uniform except for one spot (effect of tape?). Yellow spot has different thickness (23.5nm) and different osc. parameters. \\ \hline 
	250 & 3:1 & 16 & Si & 150 & 600 & RTA & 650 & 1m & N &  \\ \hline 
	\multirow{2}{*}{250} & \multirow{2}{*}{3:1} & \multirow{2}{*}{19} & Si & \multirow{2}{*}{100} & \multirow{2}{*}{400} &  &  &  & N & Replaced precursor prior to this run. \\* \cdashline{4- 4}\cdashline{7- 11}
	  &   &   & Pt &   &   &  &  &  & N &  \\ \hline 
	\multirow{2}{*}{250} & \multirow{2}{*}{3:1} & \multirow{2}{*}{20} & Si & \multirow{2}{*}{200} & \multirow{2}{*}{800} &  &  &  & N & Runs appear to be growing well.  Spectacular coloration on the reactor surface where platinum was deposited. \\* \cdashline{4- 4}\cdashline{7- 11}
	  &   &   & Pt &   &   & Fast Oven & 650 & 1m & Y &  \\ \hline 
	\multirow{2}{*}{250} & \multirow{2}{*}{3:1} & \multirow{2}{*}{21} & Si & \multirow{2}{*}{150} & \multirow{2}{*}{600} &  &  &  & N &  \\* \cdashline{4- 4}\cdashline{7- 11}
	  &   &   & Pt &   &   & Fast Oven & 650 & 1m & N & Discolored, not homogeneous across sample... \\ \hline 
	\multirow{2}{*}{250} & \multirow{2}{*}{3:1} & \multirow{2}{*}{22} & Si & \multirow{2}{*}{150} & \multirow{2}{*}{600} &  &  &  & N &  \\* \cdashline{4- 4}\cdashline{7- 10}
	  &   &   & Pt &   &   & Fast oven & 650 & 1m & Y &  \\ \hline 
	  \multicolumn{11}{c}{}\\
	  \multicolumn{11}{c}{}\\
	  \multicolumn{11}{c}{}\\
	\multirow{2}{*}{250} & \multirow{2}{*}{3:1} & \multirow{2}{*}{23} & Si & \multirow{2}{*}{200} & \multirow{2}{*}{800} &  &  &  & N &  \\* \cdashline{4- 4}\cdashline{7- 10}
	  &   &   & Pt &   &   & Fast oven & 650 & 1m & Y &  \\ \hline
	\multirow{2}{*}{250} & \multirow{2}{*}{3:1} & \multirow{2}{*}{24} & Si & \multirow{2}{*}{200} & \multirow{2}{*}{800} &  &  &  & N &   \\* \cdashline{4- 4}\cdashline{7- 10}
	  &   &   & Pt &   &   &  &  &  & N &   \\ \hline 
	\multirow{2}{*}{250} & \multirow{2}{*}{3:1} & \multirow{2}{*}{25} & Si & \multirow{2}{*}{60} & \multirow{2}{*}{240} &  &  &  & N &   \\* \cdashline{4- 4}\cdashline{7- 10}
	  &   &   & Pt &   &   &  &  &  & N &   \\ \hline 
	\multirow{2}{*}{250} & \multirow{2}{*}{2:1} & \multirow{2}{*}{26} & Si & \multirow{2}{*}{80} & \multirow{2}{*}{240} &  &  &  & N &   \\* \cdashline{4- 4}\cdashline{7- 10}
	  &   &   & Pt &   &   &  &  &  & N &   \\ \hline 
	\multirow{2}{*}{250} & \multirow{2}{*}{5:2} & \multirow{2}{*}{27} & Si & \multirow{2}{*}{110} & \multirow{2}{*}{770} &  &  &  & N &   \\* \cdashline{4- 4}\cdashline{7- 10}
	  &   &   & Pt &   &   &  &  &  & N &   \\ \hline 
	\multirow{3}{*}{250} & \multirow{3}{*}{3:1} & \multirow{3}{*}{28} & Si & \multirow{3}{*}{120} & \multirow{3}{*}{480} &  &  &  & N &   \\* \cdashline{4- 4}\cdashline{7- 10}
	  &   &   & Pt &   &   &  &  &  & N &   \\* \cdashline{4- 4}\cdashline{7- 10}
	  &   &   & STO &   &   &  &  &  & Y &  \\ \hline 
	\multirow{2}{*}{250} & \multirow{2}{*}{3:1} & \multirow{2}{*}{29} & Si & \multirow{2}{*}{50} & \multirow{2}{*}{200} &  &  &  & N &   \\* \cdashline{4- 4}\cdashline{7- 10}
	  &   &   & Pt &   &   &  &  &  & N &   \\ \hline 
	250 & 2:1 & 18 & Si & 200 & 600 &  &  &  & N & Might be running low on Pb(TMHD)2 precursor, going to add some more this afternoon \\ \hline 
	300 & 1:1 & 4 & Si & 250 & 500 &  &  &  & N &  \\ \hline 
	300 & 1:2 & 5 & Si & 200 & 600 &  &  &  & N &  \\ \hline 
	300 & 2:3 & 9 & Si & 175 & 875 &  &  &  & N & May have gotten recipe timing wrong, might have to re-do this test... \\ \hline 
	300 & 3:2 & 10 & Si & 50 & 250 &  &  &  & N &  \\ \hline 
	300 & 3:1 & 11 & Si & 150 & 450 &  &  &  & N &  \\ \hline
\end{longtable}}

\end{landscape}
%%%%%%%%%%%%%%%%%%%%%%%%%%%%%%%%%%%%%%%%%%%%%%%%%%%%
%%%%%%%%%%%%%%%%%%%%%%%%%%%%%%%%%%%%%%%%%%%%%%%%%%%%
%%%%%%%%%%%%%%%%%%%%%%%%%%%%%%%%%%%%%%%%%%%%%%%%%%%%
%\begin{landscape}
\section{ALD Reactor Diagram}
\label{sup:ALD-design}

\begin{figure}[htb]
   \centering
   \subfloat[Photograph][Photograph]{%
   	\label{fig:S100-photo}%
	\includegraphics[height=8cm]{./Figures/Appendix/ALD-schematic/savannah-s100}%
	} 
	\hspace{1cm}
  \subfloat[Schematic Diagram][Schematic Diagram]{%
   	\label{fig:S100-schematic}%
	\includegraphics[height=8cm]{./Figures/Appendix/ALD-schematic/savannah-schematic}%
	} 	
   \caption[Cambridge NanoTech, inc. S100 ALD System]%
   		{Cambridge NanoTech, inc. Savannah S100 ALD reactor. Precursors are stored in heated cylinders, flow up to the reaction zone, and byproducts are pumped out of the vacuum line on the right side. Each zone can be individually temperature controlled.}
   \label{fig:S100}
\end{figure}
%
%\begin{figure}[htbp]
%	\begin{center}
%		\includegraphics[width=0.5\textwidth]{./Figures/Appendix/ALD-schematic/savannah-s100}
%		\caption[Photograph of S100 ALD Reactor]{Photograph of the Cambridge NanoTech, inc. %
%				Savannah S100 ALD reactor}
%		\label{fig:S100-photo}
%	\end{center}
%\end{figure}
%
%\begin{figure}[htbp]
%	\begin{center}
%		\includegraphics[width=0.85\textwidth]{./Figures/Appendix/ALD-schematic/savannah-schematic}
%		\caption[Diagram of S100 ALD Reactor]{Diagram of the Cambridge NanoTech, inc. %
%				Savannah S100 ALD reactor. Precursors are stored in heated cylinders, flow up to the reaction zone, and byproducts are pumped out of the vacuum line on the right side. Each zone can be individually temperature controlled. }
%		\label{fig:S100-schematic}
%	\end{center}
%\end{figure}


%\end{landscape}
%%%%%%%%%%%%%%%%%%%%%%%%%%%%%%%%%%%%%%%%%%%%%%%%%%%%
%%%%%%%%%%%%%%%%%%%%%%%%%%%%%%%%%%%%%%%%%%%%%%%%%%%%
%%%%%%%%%%%%%%%%%%%%%%%%%%%%%%%%%%%%%%%%%%%%%%%%%%%%

\section{Recipes for S100 ALD System}
\label{sup:recipes}

Some of these recipes (\ce{HfO2}, \ce{Pt}, and \ce{TiO2}) were provided courtesy of Cambridge NanoTech, inc.\cite{CNT-web}

%%%%%%%%%%%%%%%%%%%%%%%%%%%%%%%%%%%%%%%%%%%%%%%%%%%%
%%%%%%%%%%%%%%%%%%%%%%%%%%%%%%%%%%%%%%%%%%%%%%%%%%%%
%%%%%%%%%%%%%%%%%%%%%%%%%%%%%%%%%%%%%%%%%%%%%%%%%%%%

\section{Thermal Analysis Results}
\label{sup:Thermal-Results}

\subsection{Thermogravimetric Analysis}
\label{sup:Thermal-Results-TGA}

\begin{figure}[htbp]
   \centering
   \subfloat[Mass vs. Temperature][Mass vs. Temperature]{%
   	\label{fig:TGA-HFAc-Weight}%
	\includegraphics[width=0.75\textwidth]{./Figures/Appendix/Thermal-Analysis/TGA/HFAc-Weight}%
	} \\
%	\hspace{1cm}
  \subfloat[Derivative of Mass vs. Temperature][Derivative of Mass vs. Temperature]{%
   	\label{fig:TGA-HFAc-DWeight}%
	\includegraphics[width=0.75\textwidth]{./Figures/Appendix/Thermal-Analysis/TGA/HFAc-DWeight}%
	} 	
   \caption[TGA Results for Pb(HFAc)$_{2}$ Precursor]%
   		{Plots of the results from TGA experiments on Pb(HFAc)$_{2}$. The plot shown in (a) gives the raw %
		data showing the current mass as a function of temperature. (b) gives the same data, transformed %
		to show the derivative of mass. Thus (b) shows the rate of mass loss at a given temperature. Initial %
		sample mass: 6.092 mg}
   \label{fig:TGA-HFAc}
\end{figure}

\begin{figure}[htbp]
   \centering
   \subfloat[Mass vs. Temperature][Mass vs. Temperature]{%
   	\label{fig:TGA-TMHD-Weight}%
	\includegraphics[width=0.75\textwidth]{./Figures/Appendix/Thermal-Analysis/TGA/TMHD-Weight}%
	} \\
%	\hspace{1cm}
  \subfloat[Derivative of Mass vs. Temperature][Derivative of Mass vs. Temperature]{%
   	\label{fig:TGA-TMHD-DWeight}%
	\includegraphics[width=0.75\textwidth]{./Figures/Appendix/Thermal-Analysis/TGA/TMHD-DWeight}%
	} 	
   \caption[TGA Results for Pb(HFAc)$_{2}$ Precursor]%
   		{Plots of the results from TGA experiments on Pb(TMHD)$_{2}$. As in figure~\vref{fig:TGA-HFAc}, (a) %
		presents the actual mass as a function of temperature, while (b) gives the derivative of that function. %
		Initial sample mass: 3.719 mg}
   \label{fig:TGA-TMHD}
\end{figure}


\begin{figure}[htbp]
   \centering
   \subfloat[\HFAc][\HFAc]{%
   	\label{fig:TGA-HFAc-Hold}%
	\includegraphics[width=0.75\textwidth]{./Figures/Appendix/Thermal-Analysis/TGA/HFAc-Hold}%
	} \\
%	\hspace{1cm}
  \subfloat[\TMHD][\TMHD]{%
   	\label{fig:TGA-TMHD-Hold}%
	\includegraphics[width=0.75\textwidth]{./Figures/Appendix/Thermal-Analysis/TGA/TMHD-Hold}%
	} 	
   \caption[Constant Temperature TGA Experiments]%
   		{Plots of the results from ramp-and-hold TGA experiments designed to investigate residual material %
		after complete evaporation at a given temperature. From the TGA experiments seen above (figs.~%
		\vref{fig:TGA-HFAc} and \vref{fig:TGA-TMHD}), a common temperature of 160\degC{} was chosen %
		for this experiment. Sample sizes were 3.921 mg and 4.381 mg for \HFAc{} and \TMHD{} respectively.}
   \label{fig:TGA-Hold}
\end{figure}

%%%%%%%%%%%%%%%%%%%%%%%%%%%%%

\subsection{Differential Scanning Calorimetry}
\label{sup:Thermal-Results-DSC}

\begin{figure}[htbp]
   \centering
   \subfloat[Pb(HFAc)$_{2}$][Pb(HFAc)$_{2}$]{%
   	\label{fig:DSC-HFAc}%
	\includegraphics[width=0.75\textwidth]{./Figures/Appendix/Thermal-Analysis/DSC/HFAc}%
	} \\
  \subfloat[Pb(TMHD)$_{2}$][Pb(TMHD)$_{2}$]{%
   	\label{fig:DSC-TMHD}%
	\includegraphics[width=0.75\textwidth]{./Figures/Appendix/Thermal-Analysis/DSC/TMHD}%
	} 	
   \caption[Results of DSC Experiments]%
   		{This plot shows results from DSC experiments carried out during the course of %
		this project. (a) shows the results from analyzing the Pb(HFAC)$_{2}$ compound. (b) gives %
		the data from Pb(TMHD)$_{2}$. Exothermic (heat release) flow is positive in both plots. }
   \label{fig:DSC-Data}
\end{figure}


%\begin{figure}[htbp]
%	\begin{center}
%		\includegraphics[width=0.75\textwidth]{./Figures/Appendix/Thermal-Analysis/DSC/HFAc}
%		\caption[DSC of Pb(HFAc)$_{2}$]{This plot shows the result of a DSC experiment on the %
%				Pb(HFAc)$_{2}$ precursor. It shows the major points where energy is absorbed or %
%				released, and analysis of these points can give significant information about the %
%				thermal behavior and the transition mechanisms.}
%		\label{fig:DSC-HFAc}
%	\end{center}
%\end{figure}
%
%\begin{figure}[htbp]
%	\begin{center}
%		\includegraphics[width=0.75\textwidth]{./Figures/Appendix/Thermal-Analysis/DSC/TMHD}
%		\caption[DSC of Pb(TMHD)$_{2}$]{This plot shows the result of a DSC experiment on the %
%				Pb(TMHD)$_{2}$ precursor. As in figure~\vref{fig:DSC-HFAc}, analysis of this plot %
%				can give meaningful information on what is chemically happening to the %
%				compound as the temperature changes.}
%		\label{fig:DSC-TMHD}
%	\end{center}
%\end{figure}

%%%%%%%%%%%%%%%%%%%%%%%%%%%%%%%%%%%%%%%%%%%%%%%%%%%%
%%%%%%%%%%%%%%%%%%%%%%%%%%%%%%%%%%%%%%%%%%%%%%%%%%%%
%%%%%%%%%%%%%%%%%%%%%%%%%%%%%%%%%%%%%%%%%%%%%%%%%%%%

\section{Ellipsometry Results}
\label{sup:Ellipsometry}


%%%%%%%%%%%%%%%%%%%%%%%%%%%%%%%%%%%%%%%%%%%%%%%%%%%%
%%%%%%%%%%%%%%%%%%%%%%%%%%%%%%%%%%%%%%%%%%%%%%%%%%%%
%%%%%%%%%%%%%%%%%%%%%%%%%%%%%%%%%%%%%%%%%%%%%%%%%%%%

\section{XRD Results}

