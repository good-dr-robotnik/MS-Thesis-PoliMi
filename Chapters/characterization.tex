\chapter{Material Characterization}
\label{chap:Charact}
\thispagestyle{empty}

%%%%%%%%%%%%%%%%%%%%%%%%%%%%%%%%%%%%%%%%%%%%%%%%%%%
%%%%%%%%%%%%%%%%%%%%%%%%%%%%%%%%%%%%%%%%%%%%%%%%%%%
%%%%%%%%%%%%%%%%%%%%%%%%%%%%%%%%%%%%%%%%%%%%%%%%%%%

\section{Thermal Analysis}
\label{sec:Charact-Thermal}

%%%%%%%

\subsection{Thermogravimetric Analysis}

Thermogravimetric analysis (TGA) is a very useful tool when attempting to determine the viability of a precursor in an ALD process. It allows for estimation of vaporization rate at various temperature rates as well as indications of chemical breakdown (i.e. thermalization) which would hinder the precursor's usefulness. 

At its core, TGA is a measurement of mass loss as a function of temperature or time. A small sample (1--10 mg) of material is placed in a microgram balance pan and suspended inside a furnace. The furnace is then heated at a specified rate while the sample mass is carefully monitored. For the experiments used in this study (evaluation of thermal vaporization and thermal degradation) it is important to ensure that the testing environment is inert. This is accomplished by using a platinum pan in the microgram balance and constantly purging the furnace with a small flow of dry nitrogen gas. The heating rate can be varied according to a pre-determined program to provide more information at various individual temperatures.\cite{Broido_TGA_1969,Doyle_TGA_1961,horowitz_TGA_1963,wunderlich_thermal_1990} 

This technique was used to evaluate various precursor candidates for the lead oxide half of the \PTO{} deposition procedure. The instrument used was a Q50 TGA device (fig.~\vref{fig:Q50-image}).  A detailed discussion of TGA procedures and the investigated chemicals can be found in subsequent chapters (see \vref{sec:SampFab-Precursors} and \vref{chap:Results-Thermal}). 

\begin{figure}[tb]
   \centering
   \subfloat[Q50 TGA][Q50 TGA]{%
   	\label{fig:Q50-image}%
	\includegraphics[width=0.45\linewidth]{./figures/characterization/Q50-TGA}%
	} 
  \subfloat[Q2000 DSC][Q2000 DSC]{%
   	\label{fig:Q2000-image}%
	\includegraphics[width=0.45\linewidth]{./figures/characterization/Q2000-DSC}%
	} 	
   \caption[T.A. Instruments, Inc. Instrumentation]%
   		{Photograph of the thermal analysis instrumentation used during this study, \\
		made by T.A.  Instruments, Inc.\\%
		{\tiny Image Sources: (a) \url{http://mrc.stu.edu.cn/old/Chinese/Resource/Equipments/TGA.htm} %
		via Go-Dove.com.\cite{go-dove}%
		\\%
		(b) \url{http://www.go-dove.com/en/event-16047/lot-213/TA-Instruments-Q50-Thermo-Gravimetric-Analyser} \\%
		from Semiconductor-Technology.com.\cite{Semiconductor-tec}}}
   \label{fig:TA-Instruments}
\end{figure}

%%%%%%%

\subsection{Differential Scanning Calorimetry}
	
Differential scanning calorimetry (DSC) is a technique that allows for the determination of various critical temperatures for a material, and also can highlight changes in chemical structure due to degradation or other thermal processes. 

DSC is the analysis of energy absorption as a function of temperature, which is the essence of calorimetry. DSC uses a sample and reference system to isolate the energy absorbed by the sample from that of the holder pan. Sample sizes usually range from 0.1--2 mg of material; as the samples used in this study are volatile the sample pans are hermetically sealed to prevent mass loss. The sample and reference pans are then placed inside a thermally insulated chamber. The temperature of each is carefully monitored, and differing amounts of heat are applied to negate the temperature difference between the sample and reference. The difference in absorbed heat as a function of temperature is then given as the result. In general, experiments include both heating and cooling curves to gain a complete understanding of the different energies.\cite{oneill_DSC_1964,skoog_DSC_1998,wunderlich_thermal_1990} 
	
DSC was used to analyze the behavior of precursor chemicals around their evaporation and reaction temperatures. The main goal was to determine if the material underwent any thermally-activated degradation processes at either of these two temperature ranges. At the evaporation temperature, the sample was generally cycled multiple times to simulate actual use in the ALD. These measurements were taken using a Q2000 DSC system (fig.~\vref{fig:Q2000-image}) made by T.A. Instruments, Inc.	

%%%%%%%%%%%%%%%%%%%%%%%%%%%%%%%%%%%%%%%%%%%%%%%%%%%
%%%%%%%%%%%%%%%%%%%%%%%%%%%%%%%%%%%%%%%%%%%%%%%%%%%
%%%%%%%%%%%%%%%%%%%%%%%%%%%%%%%%%%%%%%%%%%%%%%%%%%%

\section{Thin Film Characterization}
\label{sec:Charact-ThinFilm}


%%%%%%%
	
\subsection{Variable Angle Spectroscopic Ellipsometry}

Ellipsometry is a powerful non-destructive optical technique that allows for the determination of a large number of properties of complex thin film structures. The basic tenet of ellipsometry relies on the analysis of the change in polarization state of a reflected light beam after interaction with the sample. The incident beam is generally linearly polarized, but upon reflection becomes elliptically polarized due to a phase shift in the components of the beam in the s- and p-plane, as well as a change in their relative amplitudes. The phase shift is correlated to the ellipsometric parameter $\Delta$, while the amplitude change is given by $\tan\Psi$ ($\Psi$ is the angle between the s-plane and the major axis of the ellipse). The last major parameter is the incident angle, denoted by $\Phi$. A schematic diagram illustrating these parameters can be seen in figure~\vref{fig:ellipsometry}.\cite{azzam_ellipsometry_1987,schubert_infrared_2005,tompkins_spectroscopic_1999} 

\begin{figure}[tb]
   \centering
   \includegraphics[width=\linewidth]{./figures/characterization/ellipsometryDiagram_simple} 
   \caption[Ellipsometric Beam Path and Modeling Parameters]{Schematic of the beam path during an %
   					ellipsometric measurement, \\ critical parameters are indicated.\\
					{\tiny Image source: \url{http://www.tcd.ie/Physics/Surfaces/ellipsometry2.php} %
					from Trinity College of Dublin.\cite{trinity_college_dublin}}}
   \label{fig:ellipsometry}
\end{figure}

From these parameters, one can directly determine the ratio between the reflectance in the p-plane ($r_{p}$) and the reflectance in the s-plane ($r_{s}$) from the fundamental ellipsometric relation (eqn.~\vref{eq:ellipsometry}).  Once this relationship is known, the Fresnel equations (eqn.~\vref{eq:fresnel}) can be used to numerically determine the value of the complex index of refraction at the specific wavelength of the incoming beam. The complex index of refraction (eqn.~\vref{eq:complexindex}) describes the nominal index of refraction but additionally includes an imaginary term to describe absorption of light in the material (commonly referred to as the extinction coefficient, $\kappa$).\cite{azzam_ellipsometry_1987,schubert_infrared_2005,tompkins_spectroscopic_1999}  

\begin{equation}
 \label{eq:ellipsometry}
 \displaystyle
	\rho(\lambda) = \frac{\tilde{r}_{p}(\lambda)}{\tilde{r}_{s}(\lambda)} = \tan(\Psi(\lambda))e^{i\Delta(\lambda)}
\end{equation}

\begin{subequations}
\label{eq:fresnel}
\begin{align}
	r_{p}(\lambda) &= \frac{\tilde{n}_{1}(\lambda)\sqrt{1- \left(\frac{\tilde{n}_{1}(\lambda)}{\tilde{n}_{2}(\lambda)}\sin\Phi\right)} - \tilde{n}_{2}(\lambda)%
			\cos\Phi}{\tilde{n}_{1}(\lambda)\sqrt{1- \left(\frac{\tilde{n}_{1}(\lambda)}{\tilde{n}_{2}(\lambda)}\sin\Phi\right)} + %
			\tilde{n}_{2}(\lambda)\cos\Phi} \\
        	r_{s}(\lambda) &= \frac{\tilde{n}_{1}(\lambda)\cos\Phi - \tilde{n}_{2}(\lambda)\sqrt{1- \left(\frac{\tilde{n}_{1}(\lambda)}{\tilde{n}_{2}(\lambda)}\sin\Phi\right)}}%
			{\tilde{n}_{1}(\lambda)\cos\Phi + \tilde{n}_{2}(\lambda)\sqrt{1- \left(\frac{\tilde{n}_{1}(\lambda)}{\tilde{n}_{2}(\lambda)}\sin\Phi\right)}}
\end{align}
\end{subequations}

\begin{equation}
 \label{eq:complexindex}
 \displaystyle
	\tilde{n}(\lambda) = n(\lambda)+ i\kappa(\lambda)
\end{equation}

This type of analysis is sufficient for thick, isotropic samples without any surface layers (e.g. surface oxides or adsorbed gases), and can directly provide the value of $\tilde{n}$ as a function of $\lambda$. However, once layers are stacked upon one another, the system becomes very difficult to analyze directly due to interference effects between the layers, especially at varying wavelengths. It becomes necessary to use modeling techniques to determine the correct values of $\tilde{n}(\lambda)$ and thickness ($t$) for each layer.\cite{schubert_infrared_2005,tompkins_spectroscopic_1999} 

The power of ellipsometry as a high-resolution optical analysis technique stems from the use of phase and polarization changes. This allows the analysis to overcome the diffraction limit, and can be accurate down to angstroms. Properly modeling the system is critical for this analysis to be as precise as possible. Thus, there have been refinements of the ellipsometric method to greatly increase the amount of experimental data points, allowing the overall system to be over-determined and thus letting all of the systems parameters to be calculated. 
	
Variable angle spectroscopic ellipsometry (VASE) is one of these variants. Spectroscopic ellipsometry differs from single-wavelength ellipsometry by utilizing a broad-band light source as opposed to a monochromatic source. By performing ellipsometric analysis at each of the wavelengths, one can determine the wavelength (and thus photon-energy) dependence of $n$ and $\kappa$. This not only helps to improve data analysis (as it can generally be safely assumed that the values of $n$ and $\kappa$ are smooth functions of $\lambda$), but allows for the determination of many other properties of the material. Of specific importance is the complex dielectric function ($\tilde{\epsilon}$), which is related to $\tilde{n}$ by the relation shown in equation~\vref{eq:dielectricfunction}. Knowing these functions can allow for determination of electronic properties such as the bandgap energy, the absorption coefficient, amongst others. Finally, by obtaining spectra at a number of different incident angles, one directly provides additional data points across the entire wavelength spectrum. Even a small number of additional angles can quickly provide sufficient data points for the system to be over determined.\cite{azzam_ellipsometry_1987,schubert_infrared_2005,tompkins_spectroscopic_1999} 

\begin{equation}
 \label{eq:dielectricfunction}
 \displaystyle
	\tilde{\epsilon} = \epsilon_{1} + i\epsilon_{2} = \tilde{n}^{2}
\end{equation}

During this project, a VASE M-2000U system (figure~\vref{fig:M2000_image}) built by J.A. Woollam, Inc. was used to collect all of the ellipsometric data. In addition, data analysis was performed using the WVASE32{$^{\copyright}$} package also provided by J.A. Woollam, Inc. The system utilizes a rotating compensator and a CCD detector to greatly decrease data collection time by collecting data across the entire spectrum simultaneously.  More information on this system is available from the J.A. Woollam, Inc. webpage.\cite{woollam-web}

\begin{figure}[tbp]
   \centering
   \includegraphics[width=0.5\linewidth]{./figures/characterization/M2000_ellipsometer_image.png} 
   \caption[J.A. Woollam M-2000U Ellipsometer]%
   		{Photograph of the J.A. Woollam M-2000U variable \\%
   		 angle spectroscopic ellipsometer (VASE).\\%
		 {\tiny Image Source: \url{http://jawoollam.com/Gallery/m2000_manual2.html} from J.A. Woollam, Inc.%
		 \cite{woollam-web}}}
   \label{fig:M2000_image}
\end{figure}


%%%%%%%%%%%%%%%%%%%%%%%%%%%%%%%%%%%%%%%%%%%%%%%%%%%
%%%%%%%%%%%%%%%%%%%%%%%%%%%%%%%%%%%%%%%%%%%%%%%%%%%
%%%%%%%%%%%%%%%%%%%%%%%%%%%%%%%%%%%%%%%%%%%%%%%%%%%



\section{Compositional Analysis}
\label{sec:Charact-Comp}

%%%%%%%
\subsection{Energy-Dispersive X-Ray Spectroscopy}

Energy dispersive X-ray spectroscopy (EDXS) is a commonly used analysis technique for determining the composition of a sample. In this process, a sample is bombarded with high-energy electrons (2--30 keV) which interact with the sample. Some of these electrons will cause a core electron of an atom in the sample to be ejected. This leaves a vacant orbital in the inner shell, which a higher energy electron will fill. In the process of filling the vacancy, the electron will emit an X-ray photon equal to the energy difference between the two states. These energies are referred to using the common X-ray spectroscopy nomenclature (e.g. $K_{\alpha}$, $K_{\beta}$, $L_{\alpha}$). An illustration of this process can be found in figure~\vref{fig:EDXS-image}.\cite{goldstein_EDS_2003}  

Since the energies of the emitted photons are very specific to each element, the procedure can be used to identify the presence of the element in the sample. With some calibration, EDXS can also be used to quantify the relative amounts of each element in a sample using the different number of collected photons. However, this can sometimes be difficult due to some elements have overlapping spectrums as the peaks are not sharp and different elements can have similar energies for some transitions.\cite{goldstein_EDS_2003} 

One of the downsides of EDXS is that the interaction volume of electrons is much larger deeper into the sample, so the surface sensitivity of the technique is smaller than with some other techniques. Additionally, Bremsstrahlung radiation, which is caused by high-energy electrons interacting with other charged particles, provides a large degree of background noise which can drastically interfere with the techniques ability to precisely measure, or even identify, trace elements such as those in a thin film.\cite{goldstein_EDS_2003}

\begin{figure}[tb]
   \centering
   \includegraphics[width=0.66\linewidth]{./figures/characterization/EDXS-scheme} 
   \caption[Illustration of EDXS principle]%
   		{Graphic illustrating the basic mechanism for EDXS, along with the commonly used %
		notation for the various energies. The external stimulation would be a high energy %
		electron. \\{\tiny Image source: \url{http://commons.wikimedia.org/wiki/File:EDX-scheme.svg} originally %
		contributed by ``Muso'' under GFDL license.}}
   \label{fig:EDXS-image}
\end{figure}



%%%%%%%

\subsection{X-Ray Fluorescence Spectroscopy}

X-ray fluorescence spectroscopy (XRFS) is a similar technique to EDXS. In XRFS the excitation used is X-ray photons (often from a Cu K$_{\alpha}$ source with $\lambda = 1.54$ \AA), as opposed to energetic electrons. In other considerations the techniques are equivalent. 

XRFS has a lower noise floor than EDXS, due to the lack of signal from Bremsstrahlung radiation from the deceleration of electrons, allowing smaller signals to be more easily identified (such as in ultra-thin films). It does suffer the same disadvantage of having overlapping peaks. This resolution issue can be alleviated to some degree by using wavelength dispersive XRFS (WD-XRFS), which uses diffraction techniques to analyze the emitted x-ray spectrum.\cite{Vincze_XRF_2005} 

X-ray fluorescence spectroscopy was performed using a fX-SEM analysis system from iXRF Systems, Inc.\cite{iXRF-web}  It was the primary method of composition analysis for the results presented in this thesis.



%%%%%%%%%%%%%%%%%%%%%%%%%%%%%%%%%%%%%%%%%%%%%%%%%%%
%%%%%%%%%%%%%%%%%%%%%%%%%%%%%%%%%%%%%%%%%%%%%%%%%%%
%%%%%%%%%%%%%%%%%%%%%%%%%%%%%%%%%%%%%%%%%%%%%%%%%%%
	
\section{Phase Identification}
\label{sec:Charact-PhaseID}

%%%%%%%

\subsection{X-Ray Diffraction}

X-ray diffraction (XRD) is a very commonly used technique for performing this identification. Utilizing the concepts of coherent interference, which leads to Bragg's law of diffraction given in equation~\vref{eq:braggs-law}. X-rays are utilized because their wavelengths are similar to the length scales between atomic planes in crystals (1--100\AA). As the incident rays pass through the sample and are reflected by atomic planes, they can either constructively or destructively interfere.\cite{giacovazzo_XRD_1992} 

\begin{equation}
 \label{eq:braggs-law}
 \displaystyle
	n\lambda = 2d\sin\theta
\end{equation}

Figure~\vref{fig:braggs-law} gives a graphical illustration of this principle. If the extra distance traversed by photons on the first path is equal to an integer number of wavelengths there will be constructive interference (see fig.~\vref{fig:bragg-constructive}). However, if the $\theta$ angle is slightly changed the interference rapidly obtains destructive character (see fig.~\vref{fig:bragg-destructive}). In this manner, if the crystal is moved through a range of $\theta-2\theta$ values, a pattern of angles where constructive interference occurred. \cite{giacovazzo_XRD_1992}

\begin{figure}[tbp]
   \centering
   \subfloat[Constructive Interference][Constructive Interference]{%
   	\label{fig:bragg-constructive}%
	\includegraphics[width=0.475\linewidth]{./figures/characterization/bragg-diffraction-constructive}%
	} 
	\hspace{0.5cm}
  \subfloat[Destructive Interference][Destructive Interference]{%
   	\label{fig:bragg-destructive}%
	\includegraphics[width=0.475\linewidth]{./figures/characterization/bragg-diffraction-destructive}%
	} 	
   \caption[Illustration of Bragg's Law]%
   		{These two images illustrate the effect of interference on Bragg diffraction. In (a) the extra %
		path length is exactly correct to allow coherence with the other ray; this causes constructive %
		interference. (b) is the other condition, where the path length causes a phase shift of 90\Deg{}%
		and thus the rays interfere destructively.\\%
		{\tiny Image Source: \url{http://en.wikipedia.org/wiki/File:Braggs_Law.svg} originally contributed by %
		``Cdang'' under CCA-Share Alike 3.0 License}}
   \label{fig:braggs-law}
\end{figure}

The diffraction pattern can be used to determine a list of the interplane spacings (d-spacings) present in the sample. The values of peaks at various 2$\theta$ can be converted to d-spacings via the relation $d=\frac{\lambda}{2\sin\theta}$. Individual materials will have a specific pattern of reflections that allow them to be identified in the sample.\cite{giacovazzo_XRD_1992} 

This project used a Rigaku SmartLab XRD (see figure~\vref{fig:rigaku-smartlab-photo}), which utilized a Cu-K$_{\alpha}$ X-ray source ($\lambda = 1.5418$\AA).

%\begin{figure}[tbp]
%   \centering
%   \includegraphics[width=0.5\linewidth]{./figures/characterization/bruker-d8-photo.jpg} 
%   \caption[Bruker D8 Discover XRD]%
%   		{Photograph of the Bruker D8 Discover X-ray diffractometer}
%   \label{fig:bruker-d8-photo}
%\end{figure}

\begin{figure}[tbp]
   \centering
   \includegraphics[width=0.75\linewidth]{./figures/characterization/rigaku-smartlab-photo.jpg} 
   \caption[Rigaku SmartLab XRD]%
   		{Photograph of the Rigaku SmartLab X-ray diffractometer.\\%
		{\tiny Image Source: \url{http://lexi.eng.uci.edu/facilities.html} from the University of California - Irvine%
		\cite{smartlab-image}}}
   \label{fig:rigaku-smartlab-photo}
\end{figure}


%%%%%%%

%\subsection{Grazing Incidence X-Ray Diffraction}
%
%Both instruments used in this study were capable of collecting data in GIXRD mode (fixed $\omega$--2$\theta$). 


































	
	
	