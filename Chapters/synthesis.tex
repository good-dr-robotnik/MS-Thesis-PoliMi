\chapter{Synthesis Methods}
\label{chap:synthesis}
\thispagestyle{empty}

%%%%%%%%%%%%%%%%%%%%%%%%%%%%%%%%%%%%%%%%%%%%%%%%%%%
%%%%%%%%%%%%%%%%%%%%%%%%%%%%%%%%%%%%%%%%%%%%%%%%%%%
%%%%%%%%%%%%%%%%%%%%%%%%%%%%%%%%%%%%%%%%%%%%%%%%%%%

Synthesis of perovskite oxides has been demonstrated using a wide range of techniques. These range from solution-based processing methods (sol-gel approach), to physical vapor methods (molecular beam epitaxy and pulsed laser deposition), and gas phase chemical methods (chemical vapor deposition and atomic layer deposition). This review will briefly discuss sol-gel, physical vapor deposition, as well as CVD methodology, but will focus in more depth on films deposited via ALD. 

%%%%%%%%%%%%%%%%%%%%%%%%%%%%%%%%%%%%%%%%%%%%%%%%%%%
%%%%%%%%%%%%%%%%%%%%%%%%%%%%%%%%%%%%%%%%%%%%%%%%%%%
%%%%%%%%%%%%%%%%%%%%%%%%%%%%%%%%%%%%%%%%%%%%%%%%%%%

\section{Sol-Gel Processing}

\lipsum

%%%%%%%%%%%%%%%%%%%%%%%%%%%%%%%%%%%%%%%%%%%%%%%%%%%
%%%%%%%%%%%%%%%%%%%%%%%%%%%%%%%%%%%%%%%%%%%%%%%%%%%
%%%%%%%%%%%%%%%%%%%%%%%%%%%%%%%%%%%%%%%%%%%%%%%%%%%

\section{Physical Vapor Deposition}

\lipsum

%%%%%%%%%%%%%%%%%%%%%%%%%%%%%%%%%%%%%%%%%%%%%%%%%%%
%%%%%%%%%%%%%%%%%%%%%%%%%%%%%%%%%%%%%%%%%%%%%%%%%%%
%%%%%%%%%%%%%%%%%%%%%%%%%%%%%%%%%%%%%%%%%%%%%%%%%%%

\section{Metallorganic Chemical Vapor Deposition}

\lipsum

%%%%%%%%%%%%%%%%%%%%%%%%%%%%%%%%%%%%%%%%%%%%%%%%%%%
%%%%%%%%%%%%%%%%%%%%%%%%%%%%%%%%%%%%%%%%%%%%%%%%%%%
%%%%%%%%%%%%%%%%%%%%%%%%%%%%%%%%%%%%%%%%%%%%%%%%%%%

\section{Atomic Layer Deposition}
	
Atomic Layer Deposition (ALD) is a modification on standard CVD processes, with a few major differences. The defining aspect of an ALD process is the separation of the overall reaction into two steps: first the precursor is allowed to react with the substrate surface (see reaction~\ref{chem:TMA1}), excess reactant is purged from the chamber and an oxidizer is introduced to complete the reaction (see reaction~\ref{chem:TMA2}). These reactions show a very simple ALD reaction between trimethylaluminum (TMA) and water. 

\begin{reactions}
	Al(CH3)3 + M-OH_{surf} &-> M-O-Al(CH3)2_{\,surf} + CH4 \label{chem:TMA1}%
		\AddRxnDesc{TMA: Precursor-Surface Site Reaction}%
		\\
	M-O-Al(CH3)2_{\,surf} + 2H2O &-> M-O-Al(OH)2_{\,surf} + 2CH4 \label{chem:TMA2}%
		\AddRxnDesc{TMA: Precursor Ligand Oxidation \& Site Regeneration}
\end{reactions}


In this example, it is seen that the first stage allows the TMA to react with the hydrated substrate surface to form part of a layer of alumina (\ce{Al2O3}), liberating a molecule of methane as a byproduct. In the next step, the remaining ligands are stripped away from the bound TMA molecule and replacing them with hydroxy groups. This returns the system to the initial state --- where the surface is presenting sites available to react with more TMA --- and the cycle is completed. 

Having only surface reactions be permitted, as opposed to CVD where gas-phase interactions dominate, affords ALD a number of unique characteristics. One of these is the concept of the ``self-limiting'' growth mode. This behavior arises from the limited number of available reaction sites; when all of these have either been reacted with or made unavailable by a blocking mechanism such as stearic hindrance from other local chemisorbed precursor the reaction can no longer proceed. At this point, additional available precursor is not going to be utilized, and instead will be removed and treated as waste material. The system is then evacuated, and a inert purge gas such as dry nitrogen or argon (at UHP grade) is flowed through the reactor. The purge gas serves both to push any remaining gases out of the reactor as well as to help desorb physisorbed species from the surface. \reword{If these are allowed to remain adsorbed they would react with the oxidant and negate the surface-limited aspects of ALD.} The system would then again be evacuated, and the oxidant introduced and then pumped away to complete the cycle. 

In the implementation of most ALD systems, the purge gas is also used as a carrier gas for the precursors. Thus a constant flow of gas is passed through the system, instead of having it occasionally fully evacuated, and the precursor is able to be delivered from its source to the reactor more effectively. For some precursor compounds, in particular those with a low vapor pressure, having carrier-assisted transportation can greatly improve the behavior of the system. 

Because of the self-limiting behavior, each deposition cycle is limited to a theoretical maximum of one monolayer of material (in practice a much lower coverage per cycle is attained), which is far less than a unit cell. Generally per cycle growth rates range between 0.03--1.5\AA{}, with the rate being nearly invariable during most of the deposition. This gives the second defining characteristic of ALD: very high (\AA\ level) thickness resolution. The downside of this aspect is that growths are generally much slower than other types of depositions; ALD is generally slower by an order of magnitude or more than a similar CVD process, as an example. This has proved invaluable in many processes where high precision is critical, such as electronics manufacturing. Intel, for example, uses ALD to deposit extremely thin layers of a high-$\kappa$ dielectric (such as hafnia, \ce{HfO2}) for use as the gate oxide in its transistors, with layer thickness generally less than 2nm. 

This method will produce a layer of a binary oxide material (\ce{AO_{x}}), if more complex materials are desired the method must be changed. The basic principles remain the same; one would perform the procedure for depositing a cycle of a binary oxide and then change the precursor and deposit another cycle of a different oxide material. For example, if one wished to deposit \PTO{}, one would begin by depositing a layer of \ce{TiO2} and then depositing a layer of lead oxide (\ce{PbO}). Repeating this set of cycles --- a super-cycle --- would eventually form the \PTO{} film. 

However, deposition of complex oxides is not this simple in practice. In many cases, running each oxide cycle in a 1:1 ratio will deposit a non-stoichiometric material. This makes it necessary to modify the method to deposit more of one type of oxide than the other. For example, if a material is Ti-rich the super-cycle ratio would be modified to increase the number of lead oxide cycles as compared to the titania cycles. \reword{Needs more here.}

ALD reactions are rather sensitive to a number of factors, such as temperature. The temperature must be high enough that the reactants have sufficient energy to drive the surface reaction but not so high as to allow undesirable reactions to activate (e.g. precursor cracking or surface material desorption). Precursor selection is also very important, for similar reasons. The precursors must also be incapable of reacting with themselves, to allow the self-limiting mechanism to work properly. \reword{This section needs more work.}


%
%\begin{subreactions}
%	\label{chem:2TMA}
%	\begin{reactions}
%		Al(CH3)3 + M-OH_{surf} &-> M-O-Al(CH3)2_{\,surf} + CH4 \label{chem:2TMA1} \\
%		M-O-Al(CH3)2_{\,surf} + 2H2O &-> M-O-Al(OH)2_{\,surf} + 2CH4 \label{chem:2TMA2}
%	\end{reactions}
%\end{subreactions}
%
%\begin{subreactions}
%\newcommand*{\thesubequation}{(\reactiontag.\alph{equation})}
%\label{chem:TMA}
%\begin{align}
%	\cee{Al(CH3)3 + M-OH_{surface} &-> M-O-Al(CH3)2_{\ surf} + CH4}%
%		\label{chem:TMA1}
%		\\
%	\cee{M-O-Al(CH3)2_{\ surf} + 2H2O &-> M-O-Al(OH)2_{\ surf} + 2CH4}%
%		\label{chem:TMA2}
%\end{align}
%\end{subreactions}



\lipsum







