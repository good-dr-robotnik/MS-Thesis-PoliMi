\chapter{Results}
\label{ch:Results}
\thispagestyle{empty}

%%%%%%%%%%%%%%%%%%%%%%%%%%%%%%%%%%%%%%%%%%%%%%%%%%%%%%%%%%
%%%%%%%%%%%%%%%%%%%%%%%%%%%%%%%%%%%%%%%%%%%%%%%%%%%%%%%%%%
%%%%%%%%%%%%%%%%%%%%%%%%%%%%%%%%%%%%%%%%%%%%%%%%%%%%%%%%%%

As discussed, the goal of this study was to determine methods for atomic layer deposition of ferroelectric oxides. In the process of realizing this goal there were a number of different areas of study. The first is the analysis of thermal and chemical behavior of the various potential precursors, during which TGA and DSC were primarily used (see~\vref{sec:Methods-Thermal}). Secondly,  the analysis of the film growth behavior under various conditions, this was primarily measured and analyzed using the ellipsometric techniques discussed earlier (see~\vref{sec:Methods-Ellip}). Third, the film deposition needed to be tuned to produce films with a stoichiometric composition, as this was expected to produce films which would crystallize into the desired perovskite phase, see section~\vref{sec:Methods-Comp} for the methods used for this characterization. Fourth, the phase of the crystallized film was analyzed in detail to determine behavior of the films post-annealing. XRD was used extensively for this task (see~\vref{sec:Methods-XRD}).


%%%%%%%%%%%%%%%%%%%%%%%%%%%%%%%%%%%%%%%%%%%%%%%%%%%%%%%%%%
%%%%%%%%%%%%%%%%%%%%%%%%%%%%%%%%%%%%%%%%%%%%%%%%%%%%%%%%%%
%%%%%%%%%%%%%%%%%%%%%%%%%%%%%%%%%%%%%%%%%%%%%%%%%%%%%%%%%%

\section{Thermal Analysis}
\label{chap:Results-Thermal}

While a viable titanium precursor was well identified in literature as well as experimentally, as was the oxidizers that were used, there was no such universally accepted chemical used in ALD to provide a source of lead. The primary issue was either a lack of chemical stability or a undesirably low volatility in the compounds that currently were being used. TGA and DSC was performed on a number of potential candidates (see~\vref{sec:SampFab-Precursors} for more details) in order to gauge the performance of these materials. 

%%%%%%%%%%%%%%%%

\subsection{Thermogravimetric Analysis}

Thermogravimetric analysis allows for the estimation and comparison of volatility between multiple samples. It was used in this study primarily to compare properties of the two candidate precursors, \HFAc{} and \TMHD{}. Of primary consideration was the volatility of the compound --- evaluated qualitatively by analyzing the mass loss at various temperatures --- and whether or not there are indications of imperfect evaporation. The data collected for these materials can be found below. 

By analyzing the TGA curve for \HFAc{}, found in figure~\vref{fig:TGA-HFAc-Weight}, there are a number of features that are immediately noticeable. First of these is the presence of multiple stages of evaporation in the curve. These occur at approximately 170, 190, and 230\degC{}. A TGA curve for a material that is purely evaporative, e.g. pure water, will have a smooth curve. Additional steps indicate that other processes are activating, and causing changes to the compound affecting the mass loss. 

More detail can be seen by performing a derivation on the TGA curve, giving the mass loss rate. This plot can be found in figure~\vref{fig:TGA-HFAc-DWeight}. This plot shows a shoulder on the primary evaporative peak, and then another peak starting at around 170\degC{} and peaking at 190\degC{}. The shoulder indicates that even during the evaporation of the bulk of the material, before residues and other imperfections cause rate changes, a secondary mechanism is activating and impacting the mass loss rate. 

\begin{figure}[htbp]
   \centering
   \subfloat[Mass vs. Temperature][Mass vs. Temperature]{%
   	\label{fig:TGA-HFAc-Weight}%
	\includegraphics[width=0.45\textwidth]{./Figures/Data/Thermal-Analysis/TGA/HFAc-Weight}%
	}
	\hspace{1cm}
  \subfloat[Derivative of Mass vs. Temperature][Derivative of Mass vs. Temperature]{%
   	\label{fig:TGA-HFAc-DWeight}%
	\includegraphics[width=0.45\textwidth]{./Figures/Data/Thermal-Analysis/TGA/HFAc-DWeight}%
	} 	
   \caption[TGA Results for Pb(HFAc)$_{2}$ Precursor]%
   		{Plots of the results from TGA experiments on Pb(HFAc)$_{2}$. The plot shown in (a) gives the raw %
		data showing the current mass as a function of temperature. (b) gives the same data, transformed %
		to show the derivative of mass. Thus (b) shows the rate of mass loss at a given temperature. Initial %
		sample mass: 6.092 mg}
   \label{fig:TGA-HFAc}
\end{figure}

Comparing this data with that of \TMHD{}, it is immediately obvious that the evaporation mechanism for the latter is much smoother. There is no major visible step, apart from some slight changes nearing the upper end of the testing temperature range (185--200\degC{}). With a closer look at figure~\vref{fig:TGA-TMHD-DWeight}, it is easier to see that there is basically smooth vaporization up to approximately 180\degC{}, at which point the evaporation is slowed dramatically due to residue buildup. 

\begin{figure}[htbp]
   \centering
   \subfloat[Mass vs. Temperature][Mass vs. Temperature]{%
   	\label{fig:TGA-TMHD-Weight}%
	\includegraphics[width=0.45\textwidth]{./Figures/Data/Thermal-Analysis/TGA/TMHD-Weight}%
	}
	\hspace{1cm}
  \subfloat[Derivative of Mass vs. Temperature][Derivative of Mass vs. Temperature]{%
   	\label{fig:TGA-TMHD-DWeight}%
	\includegraphics[width=0.45\textwidth]{./Figures/Data/Thermal-Analysis/TGA/TMHD-DWeight}%
	} 	
   \caption[TGA Results for Pb(HFAc)$_{2}$ Precursor]%
   		{Plots of the results from TGA experiments on Pb(TMHD)$_{2}$. As in figure~\vref{fig:TGA-HFAc}, (a) %
		presents the actual mass as a function of temperature, while (b) gives the derivative of that function. %
		Initial sample mass: 3.719 mg}
   \label{fig:TGA-TMHD}
\end{figure}

Neither of these precursors evaporated cleanly, leaving residues of more than 20\% of their initial sample mass during the temperature scanning tests discussed above. When these were tested at moderate temperatures, such as those to be used for evaporation in the ALD system, both left even larger fractions of their initial masses behind. Testing at a constant temperature (160\degC{}) over a longer period of time gives the plots shown in figure~\vref{fig:TGA-Hold}. From this test, it was found that \HFAc{} left a much larger residue than \TMHD{}, 63\% and 34\% respectively. 

\begin{figure}[htbp]
   \centering
   \subfloat[\HFAc][\HFAc]{%
   	\label{fig:TGA-HFAc-Hold}%
	\includegraphics[width=0.45\textwidth]{./Figures/Data/Thermal-Analysis/TGA/HFAc-Hold}%
	}
	\hspace{1cm}
  \subfloat[\TMHD][\TMHD]{%
   	\label{fig:TGA-TMHD-Hold}%
	\includegraphics[width=0.45\textwidth]{./Figures/Data/Thermal-Analysis/TGA/TMHD-Hold}%
	} 	
   \caption[Constant Temperature TGA Experiments]%
   		{Plots of the results from ramp-and-hold TGA experiments designed to investigate residual material %
		after complete evaporation at a given temperature. From the TGA experiments seen above (figs.~%
		\vref{fig:TGA-HFAc} and \vref{fig:TGA-TMHD}), a common temperature of 160\degC{} was chosen %
		for this experiment. Sample masses were 3.921 mg and 4.381 mg for \HFAc{} and \TMHD{} respectively.}
   \label{fig:TGA-Hold}
\end{figure}

Based on the results of these tests, the lower residual mass and the cleaner evaporative process, \TMHD{} was predicted to have better performance as an ALD precursor. 

%%%%%%%%%%%%%%%%

\subsection{Differential Scanning Calorimetry}

As discussed in previous chapters, DSC is a powerful tool for analyzing the behavior of precursors. The data collected allows for the understanding of various energies in the material. 

When considering the energetic behavior of \HFAc{} (see fig.~\vref{fig:DSC-HFAc}), there are a few minor features that can be noticed. \reword{Keep going with this.}

\begin{figure}[htbp]
	\centering
	\includegraphics[width=0.66\textwidth]{./Figures/Data/Thermal-Analysis/DSC/HFAc}
	\caption[DSC Results of \HFAc{}]%
		{Plot of the DSC scan of \HFAc{}. In this plot exothermic behavior, where the sample releases heat, is %
		considered positive. Thus, the first sweep of the scan (from 0\degC{} to 200\degC{}) is negative. %
		Sample mass: 4.7 mg.}
	\label{fig:DSC-HFAc}
\end{figure}



\begin{figure}[htbp]
	\centering
	\includegraphics[width=0.66\textwidth]{./Figures/Data/Thermal-Analysis/DSC/TMHD}
	\caption[DSC Results of \TMHD{}]%
		{Plot of the DSC scan of \TMHD{}.}
	\label{fig:DSC-TMHD}
\end{figure}
\clearpage
%%%%%%%%%%%%%%%%%%%%%%%%%%%%%%%%%%%%%%%%%%%%%%%%%%%%%%%%%%
%%%%%%%%%%%%%%%%%%%%%%%%%%%%%%%%%%%%%%%%%%%%%%%%%%%%%%%%%%
%%%%%%%%%%%%%%%%%%%%%%%%%%%%%%%%%%%%%%%%%%%%%%%%%%%%%%%%%%

\section{List of Samples}
\label{chap:Results-Samples}

Based on the results from the thermal analysis, a number of samples were deposited with various deposition parameters. Information on each of these samples can be found in the following table~\vref{tbl:LoSamples}.

{
\begin{longtable}{c c c c c c c c c}
	\caption[List of Samples]{A list of samples produced during the course of this project.%
	\label{tbl:LoSamples}}\\
	\toprule
	&&&&&\multicolumn{3}{c}{Annealing}&\\ \cmidrule{6-8}
	Temp.		&Run \#	&Pb:Ti	 	&Cycles 	&Subs. 	&Type	&Temp. 		&Time &XRD\\ 
	(\degC{})		&		&Ratio		&		&Type	&		&(\degC{})	&(min) &\\ \midrule%
	\endfirsthead
	\caption[]{A list of samples used during the course of this project.}\\
	\toprule
	&&&&&\multicolumn{3}{c}{Annealing}&\\ \cmidrule{6-8}
	Temp.		&Run \#	&Pb:Ti	 	&Cycles 	&Subs. 	&Type	&Temp. 		&Time &XRD\\ 
	(\degC{})		&		&Ratio		&		&Type	&		&(\degC{})	&(min) &\\ \midrule%
	\endhead
	200	&3		&1:1		&250	&Si		&None	&N/A		&N/A		&No\\
		&2		&1:2		&250	&Si		&None	&N/A		&N/A		&No\\
		&30		&3:1		&160	&Si		&None	&N/A		&N/A		&No\\
		&		&		&		&Pt-Si	&None	&N/A		&N/A		&No\\ \midrule
	250	&0		&1:1		&625	&Si		&Oven	&650	&120	&Yes\\
		&		&		&		&		&Oven	&900	&120	&Yes\\
		&		&		&		&		&RTA	&900	&10		&No\\
		&1		&1:1		&475	&Si		&None	&N/A		&N/A		&No\\
		&6		&1:2		&250	&Si		&None 	&N/A		&N/A		&No\\
		&13		&3:1		&250	&Si		&None 	&N/A		&N/A		&No\\
		&16		&3:1		&150	&Si		&RTA	&650	&1		&No\\
		&19		&3:1		&100	&Si		&None 	&N/A		&N/A		&No\\
		&		&		&		&Pt-Si	&None 	&N/A		&N/A		&No\\
		&20		&3:1		&200	&Si		&None 	&N/A		&N/A		&No\\
		&		&		&		&Pt-Si	&Oven	&650	&90		&Yes\\
		&		&		&		&STO	&Oven	&650	&90		&No\\
		&21		&3:1		&150	&Si		&None 	&N/A		&N/A		&No\\
		&		&		&		&Pt-Si	&Oven	&650	&90		&No\\
		&		&		&		&STO	&Oven	&650	&90		&No\\
		&22		&3:1		&150	&Si		&None 	&N/A		&N/A		&No\\
		&		&		&		&Pt-Si	&Oven	&650	&90		&Yes\\
		&23		&3:1		&200	&Si		&None 	&N/A		&N/A		&No\\
		&		&		&		&Pt-Si	&Oven	&650	&90		&Yes\\
		&28		&3:1		&120	&STO	&Oven	&650	&90		&Yes\\
	\bottomrule
\end{longtable}}

%%%%%%%%%%%%%%%%%%%%%%%%%%%%%%%%%%%%%%%%%%%%%%%%%%%%%%%%%%
%%%%%%%%%%%%%%%%%%%%%%%%%%%%%%%%%%%%%%%%%%%%%%%%%%%%%%%%%%
%%%%%%%%%%%%%%%%%%%%%%%%%%%%%%%%%%%%%%%%%%%%%%%%%%%%%%%%%%

\section{Ellipsometry}
\label{chap:Results-Ellipsometry}



%%%%%%%%%%%%%%%%%%%%%%%%%%%%%%%%%%%%%%%%%%%%%%%%%%%%%%%%%%
%%%%%%%%%%%%%%%%%%%%%%%%%%%%%%%%%%%%%%%%%%%%%%%%%%%%%%%%%%
%%%%%%%%%%%%%%%%%%%%%%%%%%%%%%%%%%%%%%%%%%%%%%%%%%%%%%%%%%

\section{Composition}
\label{chap:Results-Composition}



Be sure to note the issues with composition measurements of PTO on STO. Surface limited/depth sensitive measurements required but not readily available. 

\begin{table}[htbp]
	\centering
	\caption[XRF Calculated Compositions]{Calculated compositions of selected samples, determined via XRF. \\Composition percentages are all $\pm$1\%.\label{tbl:XRF-compositions}}
	\begin{tabular}{l l r r r}
	\toprule
	&&\multicolumn{3}{c}{Composition (\%)}\\
	\cmidrule{3-5}
	Run \#&Substrate&Lead&Titanium&Ti:Pb Ratio\\
	\midrule
% 	Run	Sub-Type		Pb%		Ti%		Ti:Pb ratio
	0	&\ce{SiO2}	&55.99	&44.01	&0.786\\
	1	&\ce{SiO2}	&55.00	&45.00	&0.809\\
	13	&\ce{SiO2}	&53.96	&46.04	&0.853\\
	16	&\ce{SiO2}	&49.45	&50.55	&1.022\\
	19	&\ce{SiO2}	&65.87	&34.13	&0.518\\
		&Pt-Si		&42.86	&57.14	&1.333\\
	20	&\ce{SiO2}	&56.52	&43.48	&0.769\\
		&Pt-Si		&51.43	&48.57	&0.944\\
	21	&\ce{SiO2}	&69.60	&30.40	&0.437\\
		&Pt-Si		&56.08	&43.92	&0.783\\
	22	&\ce{SiO2}	&67.64	&32.36	&0.478\\
		&Pt-Si		&56.06	&43.94	&0.784\\
	23	&\ce{SiO2}	&66.89	&33.11	&0.495\\
		&Pt-Si		&49.06	&50.94	&1.038\\
	24	&\ce{SiO2}	&68.96	&31.06	&0.450\\
		&Pt-Si		&62.16	&37.84	&0.609\\
	\bottomrule
	\end{tabular}
\end{table}

%%%%%%%%%%%%%%%%%%%%%%%%%%%%%%%%%%%%%%%%%%%%%%%%%%%%%%%%%%
%%%%%%%%%%%%%%%%%%%%%%%%%%%%%%%%%%%%%%%%%%%%%%%%%%%%%%%%%%
%%%%%%%%%%%%%%%%%%%%%%%%%%%%%%%%%%%%%%%%%%%%%%%%%%%%%%%%%%


\section{X-Ray Diffraction}
\label{chap:Results-XRD}

\lipsum








