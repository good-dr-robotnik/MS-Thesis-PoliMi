\chapter{Results}
\label{ch:Results}
\thispagestyle{empty}

%%%%%%%%%%%%%%%%%%%%%%%%%%%%%%%%%%%%%%%%%%%%%%%%%%%%%%%%%%
%%%%%%%%%%%%%%%%%%%%%%%%%%%%%%%%%%%%%%%%%%%%%%%%%%%%%%%%%%
%%%%%%%%%%%%%%%%%%%%%%%%%%%%%%%%%%%%%%%%%%%%%%%%%%%%%%%%%%

As discussed, the goal of this study was to determine methods for atomic layer deposition of ferroelectric oxides. In the process of realizing this goal there were a number of different areas of study. The first is the analysis of thermal and chemical behavior of the various potential precursors, during which TGA and DSC were primarily used (see~\vref{sec:Methods-TGA} and \vref{sec:Methods-DSC}). Secondly,  the analysis of the film growth behavior under various conditions, this was primarily measured and analyzed using the ellipsometric techniques discussed earlier (see~\vref{sec:Methods-Ellip}). Third, the film deposition needed to be tuned to produce films with a stoichiometric composition, as this was expected to produce films which would crystallize into the desired perovskite phase, see section~\vref{sec:Methods-Comp} for the methods used for this characterization. Fourth, the phase of the crystallized film was analyzed in detail to determine behavior of the films post-annealing. XRD was used extensively for this task (see~\vref{sec:Methods-XRD}).


%%%%%%%%%%%%%%%%%%%%%%%%%%%%%%%%%%%%%%%%%%%%%%%%%%%%%%%%%%
%%%%%%%%%%%%%%%%%%%%%%%%%%%%%%%%%%%%%%%%%%%%%%%%%%%%%%%%%%
%%%%%%%%%%%%%%%%%%%%%%%%%%%%%%%%%%%%%%%%%%%%%%%%%%%%%%%%%%

\section{Thermal Analysis}
\label{chap:Results-Thermal}

While a viable titanium precursor was well identified in literature as well as experimentally, as was the oxidizers that were used, there was no such universally accepted chemical used in ALD to provide a source of lead. The primary issue was either a lack of chemical stability or a undesirably low volatility in the compounds that currently were being used. TGA and DSC was performed on a number of potential candidates (see~\vref{sec:SampFab-Precursor} for more details) in order to gauge the performance of these materials. 

%%%%%%%%%%%%%%%%

\subsection{Thermogravimetric Analysis}


%%%%%%%%%%%%%%%%

\subsection{Differential Scanning Calorimetry}

%%%%%%%%%%%%%%%%%%%%%%%%%%%%%%%%%%%%%%%%%%%%%%%%%%%%%%%%%%
%%%%%%%%%%%%%%%%%%%%%%%%%%%%%%%%%%%%%%%%%%%%%%%%%%%%%%%%%%
%%%%%%%%%%%%%%%%%%%%%%%%%%%%%%%%%%%%%%%%%%%%%%%%%%%%%%%%%%

\section{Ellipsometry}
\label{chap:Results-Ellipsometry}

\lipsum

%%%%%%%%%%%%%%%%%%%%%%%%%%%%%%%%%%%%%%%%%%%%%%%%%%%%%%%%%%
%%%%%%%%%%%%%%%%%%%%%%%%%%%%%%%%%%%%%%%%%%%%%%%%%%%%%%%%%%
%%%%%%%%%%%%%%%%%%%%%%%%%%%%%%%%%%%%%%%%%%%%%%%%%%%%%%%%%%

\section{Composition}
\label{chap:Results-Composition}

\lipsum

%%%%%%%%%%%%%%%%%%%%%%%%%%%%%%%%%%%%%%%%%%%%%%%%%%%%%%%%%%
%%%%%%%%%%%%%%%%%%%%%%%%%%%%%%%%%%%%%%%%%%%%%%%%%%%%%%%%%%
%%%%%%%%%%%%%%%%%%%%%%%%%%%%%%%%%%%%%%%%%%%%%%%%%%%%%%%%%%


\section{X-Ray Diffraction}
\label{chap:Results-XRD}

\lipsum








