\chapter{Results}
\label{ch:Results}
\thispagestyle{empty}

%%%%%%%%%%%%%%%%%%%%%%%%%%%%%%%%%%%%%%%%%%%%%%%%%%%%%%%%%%
%%%%%%%%%%%%%%%%%%%%%%%%%%%%%%%%%%%%%%%%%%%%%%%%%%%%%%%%%%
%%%%%%%%%%%%%%%%%%%%%%%%%%%%%%%%%%%%%%%%%%%%%%%%%%%%%%%%%%

As discussed, the goal of this study was to determine methods for atomic layer deposition of ferroelectric oxides. In the process of realizing this goal there were a number of different areas of study. The first is the analysis of thermal and chemical behavior of the various potential precursors, during which TGA and DSC were primarily used (see~\vref{sec:Methods-Thermal}). Secondly,  the analysis of the film growth behavior under various conditions, this was primarily measured and analyzed using the ellipsometric techniques discussed earlier (see~\vref{sec:Methods-Ellip}). Third, the film deposition needed to be tuned to produce films with a stoichiometric composition, as this was expected to produce films which would crystallize into the desired perovskite phase, see section~\vref{sec:Methods-Comp} for the methods used for this characterization. Fourth, the phase of the crystallized film was analyzed in detail to determine behavior of the films post-annealing. XRD was used extensively for this task (see~\vref{sec:Methods-XRD}).


%%%%%%%%%%%%%%%%%%%%%%%%%%%%%%%%%%%%%%%%%%%%%%%%%%%%%%%%%%
%%%%%%%%%%%%%%%%%%%%%%%%%%%%%%%%%%%%%%%%%%%%%%%%%%%%%%%%%%
%%%%%%%%%%%%%%%%%%%%%%%%%%%%%%%%%%%%%%%%%%%%%%%%%%%%%%%%%%

\section{Thermal Analysis}
\label{chap:Results-Thermal}

While a viable titanium precursor was well identified in literature as well as experimentally, as was the oxidizers that were used, there was no such universally accepted chemical used in ALD to provide a source of lead. The primary issue was either a lack of chemical stability or a undesirably low volatility in the compounds that currently were being used. TGA and DSC was performed on a number of potential candidates (see~\vref{sec:SampFab-Precursors} for more details) in order to gauge the performance of these materials. 

%%%%%%%%%%%%%%%%

\subsection{Thermogravimetric Analysis}

Thermogravimetric analysis allows for the estimation and comparison of volatility between multiple samples. It was used in this study to compare \HFAc{} and \TMHD{}, primarily their individual volatilities --- by analyzing the mass loss at various temperatures --- and whether or not there are indications of imperfect evaporation. The data collected for these materials can be found in section~\vref{sup:Thermal-Results-TGA}. 

By analyzing the TGA curve for \HFAc{}, found in figure~\vref{fig:TGA-HFAc-Weight}, there are a number of features that are immediately noticeable. First of these is the multiple stages of evaporation present in the curve. These occur at approximately 170, 190, and 230\degC{}. A TGA curve for a material that is purely evaporative, e.g. pure water, will have a single step in the curve. Additional steps indicate that other processes are activating, and causing changes to the compound affecting the mass loss. 

More detail can be seen by performing a derivation on the TGA curve, giving the mass loss rate. This plot can be found in figure~\vref{fig:TGA-HFAc-DWeight}. This plot shows a shoulder on the primary evaporative peak, and then another peak starting at around 170\degC{} and peaking at 190\degC{}. The shoulder indicates that even during the evaporation of most of the material, a secondary mechanism is activating causing additional mass loss. 



%%%%%%%%%%%%%%%%

\subsection{Differential Scanning Calorimetry}
\clearpage
%%%%%%%%%%%%%%%%%%%%%%%%%%%%%%%%%%%%%%%%%%%%%%%%%%%%%%%%%%
%%%%%%%%%%%%%%%%%%%%%%%%%%%%%%%%%%%%%%%%%%%%%%%%%%%%%%%%%%
%%%%%%%%%%%%%%%%%%%%%%%%%%%%%%%%%%%%%%%%%%%%%%%%%%%%%%%%%%

\section{List of Samples}
\label{chap:Results-Samples}

Based on the results from the thermal analysis, a number of samples were deposited with various deposition parameters. Information on each of these samples can be found in the following table~\vref{tbl:LoSamples}.

{
\begin{longtable}{c c c c c c c c c}
	\caption[List of Samples]{A list of samples produced during the course of this project.%
	\label{tbl:LoSamples}}\\
	\toprule
	&&&&&\multicolumn{3}{c}{Annealing}&\\ \cmidrule{6-8}
	Temp.		&Run \#	&Pb:Ti	 	&Cycles 	&Subs. 	&Type	&Temp. 		&Time &XRD\\ 
	(\degC{})		&		&Ratio		&		&Type	&		&(\degC{})	&(min) &\\ \midrule%
	\endfirsthead
	\caption[]{A list of samples used during the course of this project.}\\
	\toprule
	&&&&&\multicolumn{3}{c}{Annealing}&\\ \cmidrule{6-8}
	Temp.		&Run \#	&Pb:Ti	 	&Cycles 	&Subs. 	&Type	&Temp. 		&Time &XRD\\ 
	(\degC{})		&		&Ratio		&		&Type	&		&(\degC{})	&(min) &\\ \midrule%
	\endhead
	200	&3		&1:1		&250	&Si		&None	&N/A		&N/A		&No\\
		&2		&1:2		&250	&Si		&None	&N/A		&N/A		&No\\
		&30		&3:1		&160	&Si		&None	&N/A		&N/A		&No\\
		&		&		&		&Pt-Si	&None	&N/A		&N/A		&No\\ \midrule
	250	&0		&1:1		&625	&Si		&Oven	&650	&120	&Yes\\
		&		&		&		&		&Oven	&900	&120	&Yes\\
		&		&		&		&		&RTA	&900	&10		&No\\
		&1		&1:1		&475	&Si		&None	&N/A		&N/A		&No\\
		&6		&1:2		&250	&Si		&None 	&N/A		&N/A		&No\\
		&13		&3:1		&250	&Si		&None 	&N/A		&N/A		&No\\
		&16		&3:1		&150	&Si		&RTA	&650	&1		&No\\
		&19		&3:1		&100	&Si		&None 	&N/A		&N/A		&No\\
		&		&		&		&Pt-Si	&None 	&N/A		&N/A		&No\\
		&20		&3:1		&200	&Si		&None 	&N/A		&N/A		&No\\
		&		&		&		&Pt-Si	&Oven	&650	&90		&Yes\\
		&		&		&		&STO	&Oven	&650	&90		&No\\
		&21		&3:1		&150	&Si		&None 	&N/A		&N/A		&No\\
		&		&		&		&Pt-Si	&Oven	&650	&90		&No\\
		&		&		&		&STO	&Oven	&650	&90		&No\\
		&22		&3:1		&150	&Si		&None 	&N/A		&N/A		&No\\
		&		&		&		&Pt-Si	&Oven	&650	&90		&Yes\\
		&23		&3:1		&200	&Si		&None 	&N/A		&N/A		&No\\
		&		&		&		&Pt-Si	&Oven	&650	&90		&Yes\\
		&28		&3:1		&120	&STO	&Oven	&650	&90		&Yes\\
	\bottomrule
\end{longtable}}

%%%%%%%%%%%%%%%%%%%%%%%%%%%%%%%%%%%%%%%%%%%%%%%%%%%%%%%%%%
%%%%%%%%%%%%%%%%%%%%%%%%%%%%%%%%%%%%%%%%%%%%%%%%%%%%%%%%%%
%%%%%%%%%%%%%%%%%%%%%%%%%%%%%%%%%%%%%%%%%%%%%%%%%%%%%%%%%%

\section{Ellipsometry}
\label{chap:Results-Ellipsometry}



%%%%%%%%%%%%%%%%%%%%%%%%%%%%%%%%%%%%%%%%%%%%%%%%%%%%%%%%%%
%%%%%%%%%%%%%%%%%%%%%%%%%%%%%%%%%%%%%%%%%%%%%%%%%%%%%%%%%%
%%%%%%%%%%%%%%%%%%%%%%%%%%%%%%%%%%%%%%%%%%%%%%%%%%%%%%%%%%

\section{Composition}
\label{chap:Results-Composition}



Be sure to note the issues with composition measurements of PTO on STO. Surface limited/depth sensitive measurements required but not readily available. 

%%%%%%%%%%%%%%%%%%%%%%%%%%%%%%%%%%%%%%%%%%%%%%%%%%%%%%%%%%
%%%%%%%%%%%%%%%%%%%%%%%%%%%%%%%%%%%%%%%%%%%%%%%%%%%%%%%%%%
%%%%%%%%%%%%%%%%%%%%%%%%%%%%%%%%%%%%%%%%%%%%%%%%%%%%%%%%%%


\section{X-Ray Diffraction}
\label{chap:Results-XRD}

\lipsum








