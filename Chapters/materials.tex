\chapter{Lead Titanate}
\label{chap:\thechapter}
\thispagestyle{empty}


%%%%%%%%%%%%%%%%%%%%%%%%%%%%%%%%%%%%%%%%%%%%%%%%%%%
%%%%%%%%%%%%%%%%%%%%%%%%%%%%%%%%%%%%%%%%%%%%%%%%%%%
%%%%%%%%%%%%%%%%%%%%%%%%%%%%%%%%%%%%%%%%%%%%%%%%%%%

\section{Structure}

Lead titanate (\PTO{}, PTO) naturally orders into the tetragonal perovskite crystal structure at room temperature (figure~\ref{fig:PTO-crystal}). The structure can be affected by compositional changes, temperature, or strain (primarily in thin-film systems), allowing a transition to a cubic phase. In the perovskite crystal structure, the central cation (\TiIon{} in the case of \PTO{}) is encapsulated in a octahedral cage of anions (\OIon{}), with the remaining cations (\PbIon{}) situated in the eight corners of the unit cell. if the material was doped (as in a mixed solid-solution), some of the cations would be replaced with the dopant ions, for example \ZrIon{} would be randomly distributed in  \TiIon{} sites in the \PZT{} (PZT) system.

\begin{figure}[htbp]
   \begin{center}
   \includegraphics[scale=1.25]{./figures/materials/pbtio3-crystal.pdf} % requires the graphicx package
   \caption[Tetragonal perovskite structure of \PTO{}]{Tetragonal perovskite structure of \PTO{}. \\Grey, red, and blue spheres refer to \PbIon{}, %
   		\TiIon{}, or \OIon{}, respectively}
   \label{fig:PTO-crystal}
   \end{center}
\end{figure}

%%%%%%%%%

\subsection{Effect of Temperature}

The transition from tetragonal to cubic perovskite is highly dependent on temperature. The critical temperature at which this transition occurs is referred to as the Curie temperature (\Tc{}). If the material cools through this temperature, a lengthening of the `c' axis of the unit cell spontaneously occurs. This creates anisotropy in the structure and allows for an anisotropic charge distribution to develop. In lead titanate this is caused by the shifting of the titanium ion, along with a slight shift of some of the oxygen ions as well (visible in figure~\ref{fig:PTO-crystal}). Thus, a permanent dipole is created whose magnitude increases as the system cools further from \Tc{}. This permanent dipole allows the system to exhibit ferroelectricity, implying an ability to semi-permanently switch the orientation of the dipole in the material. This switching can be reversed, but this will not occur spontaneously. 

%%%%%%%%%%%%%%%%%%%%%%%%%%%%%%%%%%%%%%%%%%%%%%%%%%%
%%%%%%%%%%%%%%%%%%%%%%%%%%%%%%%%%%%%%%%%%%%%%%%%%%%
%%%%%%%%%%%%%%%%%%%%%%%%%%%%%%%%%%%%%%%%%%%%%%%%%%%


\section{Ferroelectricity}

\lipsum

%%%%%%%%%%%%%%%%%%%%%%%%%%%%%%%%%%%%%%%%%%%%%%%%%%%
%%%%%%%%%%%%%%%%%%%%%%%%%%%%%%%%%%%%%%%%%%%%%%%%%%%
%%%%%%%%%%%%%%%%%%%%%%%%%%%%%%%%%%%%%%%%%%%%%%%%%%%

