\chapter{Sample Fabrication}
\label{ch:Fabrication}
\thispagestyle{empty}


%%%%%%%%%%%%%%%%%%%%%%%%%%%%%%%%%%%%%%%%%%%%%%%%%%%%%%%%%%
%%%%%%%%%%%%%%%%%%%%%%%%%%%%%%%%%%%%%%%%%%%%%%%%%%%%%%%%%%
%%%%%%%%%%%%%%%%%%%%%%%%%%%%%%%%%%%%%%%%%%%%%%%%%%%%%%%%%%

\section{Precursor Selection}

\lipsum

%%%%%%%%%%%%%%%
\subsection{Titanium Source}

The source of titanium that was used was titanium isopropoxide (\TiOiPr{}). This compound is very commonly used for ALD techniques 

%%%%%%%%%%%%%%%
\subsection{Lead Source}

\lipsum

%%%%%%%%%%%%%%%%%%%%%%%%%%%%%%%%%%%%%%%%%%%%%%%%%%%%%%%%%%
%%%%%%%%%%%%%%%%%%%%%%%%%%%%%%%%%%%%%%%%%%%%%%%%%%%%%%%%%%
%%%%%%%%%%%%%%%%%%%%%%%%%%%%%%%%%%%%%%%%%%%%%%%%%%%%%%%%%%

\section{Substrate Preparation}

Fabrication and preparation of substrates was an important part of the deposition process. Some substrates were purchased and simply cleaned, others needed to be fabricated or otherwise processed prior to cleaning and use in deposition. Three main types of substrates were used: thermally oxidized single crystal silicon (100) wafers, silicon wafers that have a thin layer of platinum deposited on the surface, and strontium titanate (100) single crystal substrates. 

%%%%%%%%%%%%%%%
\subsection{Si(100)} \label{sect:Si}

The silicon substrates were prepared simply. 4'' silicon wafers with 200 nm of thermally grown oxide were diced into 1.5 cm x 1.5 cm pieces. When a sample was to be used for deposition, it was cleaned by sonicating in acetone, isopropanol (each for one minute), and subsequently 5 minutes of sonication in deionized (DI) water. These were then air dried with dry nitrogen. Finally, the substrates were cleaned in a oxygen plasma cleaning system to remove any remaining organic residues. 


%%%%%%%%%%%%%%%

\subsection{Platinized Si(100)}

Platinized silicon substrates were prepared in a similar manner to the Si(100) samples. For the initial platinization, a large piece of silicon wafer (5 x 5 cm$^{2}$) was prepared in the manner shown above. Then a 15 nm layer of platinum was deposited via ALD (deposition recipe can be found in \ref{sup:recipes}). The substrates were then cleaved into smaller pieces for later use. 

If the samples are stored, it is recommended to again clean the samples in the standard procedure (see \ref{sect:Si}).

%%%%%%%%%%%%%%%

\subsection{STO(100) and Nb:STO(100)}

Oxide crystal substrates were prepared in such a fashion as to promote the formation of atomically flat terraces. This has the advantage of promoting a uniform surface species across the entire sample --- the etchant used in this process leaves the sample titania-terminated. 

To prepare these samples, the substrates are first pre-cleaned in a three step sonication process. The samples were cleaned for five minutes in each of acetone, methanol, and isopropyl alcohol. Subsequently, the samples were sonicated for fifteen minutes in DI water. \reword{Need to find reference and exact timings}. The substrates were then dipped into buffered hydrofluoric acid to etch for \reword{35 seconds}, then removed and flushed with copious DI water.  Once the sample is thoroughly rinsed, the samples are dried using dry nitrogen. 

After the etching process, the samples are annealed at 1050\degC{} for \reword{two hours}. Once the samples are cooled, they are ready for immediate use. AFM can be used to confirm the presence of atomic terraces. 

If the samples are stored, it is recommended to again clean the samples in the standard procedure (see \ref{sect:Si}).

%%%%%%%%%%%%%%%%%%%%%%%%%%%%%%%%%%%%%%%%%%%%%%%%%%%%%%%%%%
%%%%%%%%%%%%%%%%%%%%%%%%%%%%%%%%%%%%%%%%%%%%%%%%%%%%%%%%%%
%%%%%%%%%%%%%%%%%%%%%%%%%%%%%%%%%%%%%%%%%%%%%%%%%%%%%%%%%%

\section{Deposition Parameters}

There are four main parameters that can affect the behavior of an ALD deposition. 

%%%%%%%%%%%%%%%

\subsection{Growth Temperature}

The temperature of the growth chamber has a strong effect on reaction behavior. ALD reactions are sensitive to temperature, and will only proceed properly within a certain range known as the `ALD window.' Outside of this range, the reaction enters one of a number of different regimes; these are determined by comparing the growth rate of the deposition to that of a reaction in the self-limiting saturated ``ALD mode.'' 

If the growth temperature is less than the lower bound of the ALD window, the two regimes are condensation limited and activation energy limited. Condensation limited growth occurs when the substrate temperature is low enough that precursor condenses onto the surface without reacting with the presented sites. This causes higher than expected growth rates, and a lack of self-limiting behavior. If the reaction instead proceeds into the activation energy limited regime, molecules of precursor lack sufficient energy to react with the surface. This is characterized by lower deposition rates. 

Conversely, if the reactor temperature is excessive the reaction again become anomalous. Decomposition limited growth, characterized by excessive deposition, is a result of thermal cracking of the precursor materials. This reaction is not limited to the surface, and accounts for the extra material being deposited. Lower deposition rates indicate that the temperature is sufficient to cause desorption of previously-reacted material from the sample. 

For an ALD run to be successful, the acceptable temperature window for all of the reactions should overlap in some temperature range. This can become difficult with reactions requiring multiple metal precursors (e.g. \PTO, a combination of \ce{TiO2} and \ce{PbO}), as these can have widely varying ALD windows for their respective reactions. 

%%%%%%%%%%%%%%%

\subsection{Precursor Dosage}



%%%%%%%%%%%%%%%

\subsection{Purge Time}

\lipsum

%%%%%%%%%%%%%%%

\subsection{Exposure Time}

\lipsum

%%%%%%%%%%%%%%%%%%%%%%%%%%%%%%%%%%%%%%%%%%%%%%%%%%%%%%%%%%
%%%%%%%%%%%%%%%%%%%%%%%%%%%%%%%%%%%%%%%%%%%%%%%%%%%%%%%%%%
%%%%%%%%%%%%%%%%%%%%%%%%%%%%%%%%%%%%%%%%%%%%%%%%%%%%%%%%%%

\section{Post-Deposition Annealing}

Two types of annealing procedures were used in this study. Oven annealing, with the simple use of a furnace in ambient atmosphere; and rapid thermal annealing (RTA), characterized by very high heating and cooling rates and performed in an inert atmosphere (dry \ce{N2}). 


%%%%%%%%%%%%%%%

\subsection{Oven Annealing}



%%%%%%%%%%%%%%%

\subsection{Rapid Thermal Annealing}







