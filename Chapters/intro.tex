\chapter{Introduction}
\label{ch:Intro}
\thispagestyle{empty}


%%%%%%%%%%%%%%%%%%%%%%%%%%%%%%%%%%%%%%%%%%%%%%%%%%%
%%%%%%%%%%%%%%%%%%%%%%%%%%%%%%%%%%%%%%%%%%%%%%%%%%%
%%%%%%%%%%%%%%%%%%%%%%%%%%%%%%%%%%%%%%%%%%%%%%%%%%%

Modern technology stands on the shoulders of modern materials, and the two are inextricably linked together. Whenever a new material or property has been discovered or engineered, applications of this new capability lie just over the horizon. 

One of the hotbeds of innovation and interest in novel applications of material properties is the field of oxide chemistry. Such a seemingly simple class of materials, examples of which are two of the most abundant compounds on Earth (i.e. silica and alumina), has an incredibly rich set of capabilities derived from the incredible variety of potential materials. One example is the field of high-\textsc{k} dielectric oxides, such as hafnium oxide (\ce{HfO2}) and zirconium oxide (\ce{ZrO2}),\cite{chen_atomic_2007,fischer_batch_2008} which are allowing the semiconductor industry to produce more capable devices while reducing power draw. Ferromagnetic and antiferromagnetic oxides form the basis of the ubiquitous magnetic hard disk drive, advancements in deposition and microstructure control allow for the steady increase in capacities that consumers have become accustomed to.\cite{scheffe_atomic_2009}

The class of ferroelectric oxides comprises the family of materials that this thesis will consider. One material in particular will be the primary subject: lead titanate (PTO, \PTO{}).\cite{harjuoja_2006,lee_effects_2009} Lead titanate is one end group of the lead zirconate titanate family of ferroelectric oxides (PZT, \ce{PbZr_{1-x}Ti_{x}O3}).\cite{Muralt_2000} The PZT family is of particular technical importance to many applications, especially the \ce{PbZr_{0.52}Ti_{0.48}O3} form which exhibits exemplary piezoelectric, ferroelectric, and pyroelectric behavior. PZT's piezoelectric capabilities allow for it to be used as actuators and sensors in innumerable devices (e.g. tip actuators in AFM microscopes and transducers in ultrasound imagers). 

Other applications of such materials are found when they are created as nanoscale thin films, where these capabilities are both more strongly exhibited. One of the more prevalent methods of producing nanoscale films is chemical vapor deposition (CVD), a powerful process that has been used to deposit a vast number of material types and structures.\cite{Matsumara_1998}

However, there are disadvantages to the CVD method of film deposition. Great care must be taken to obtain highly regular film thicknesses, and the method of producing most oxide films --- a variation called metallorganic CVD (MOCVD) after the metallorganic compounds used as reactants --- allows a significant amount of hazardous material to be released into the environment as toxic and nano-particulate byproducts. 

Another deposition method, atomic layer deposition (ALD), is becoming popular in industry applications as an alternative to CVD. ALD provides the user with ultra-high resolution on film thickness, a lower operating temperature than nearly all CVD processes, amongst other features. In addition, due to the nature of the process, ALD is capable of consuming far less precursor material as well as utilizing a far larger percentage in depositing the film as opposed to producing the types of byproducts caused by CVD.\cite{ALD-Handbook} This makes ALD, in situations where it can be applied, the more economical choice, as well as being more sustainable and environmentally friendly. To this end, the National Science Foundation (NSF) has awarded funding (under 2012 grant \#1200940) toward the development of alternative ALD precursors, models, and processes that will attempt to provide additional improvements to the sustainability of ALD.

However, well described processes for producing ALD films of many materials have not been developed. Binary oxides (\ce{A_{x}O_{y}}) have been explored in a fair amount of depth,\cite{chen_atomic_2007,de_ridder_growth_2002,knez_atomic_2006,lee_al2o3_2003,scheffe_atomic_2009} but there is less research work available covering processes for depositing complex oxide films via ALD. 

%%%%%%%%%%%%%%%%%%%%%%%%%%%%%%%%%%%%%%%%%%%%%%%%%%
%%%%%%%%%%%%%%%%%%%%%%%%%%%%%%%%%%%%%%%%%%%%%%%%%%
%%%%%%%%%%%%%%%%%%%%%%%%%%%%%%%%%%%%%%%%%%%%%%%%%%

\section{Project Scope}
\label{sec:Intro-Scope}

This thesis will cover the main steps taken in developing a thin film deposition processes using ALD. In chapter~\ref{chap:Materials} lead titanate as a material will be introduced, discussing its structure and some of the desirable properties that the oxide exhibits due to this structure. 

Chapter~\ref{chap:Synth} will be devoted to a few of the commonly used methods used to produce films of complex oxides, sol-gel and MOCVD, as well as introducing in detail the primary focus of this project: the deposition of thin films via ALD. 

Following this, chapter~\ref{ch:SampFab} will go into detail of the various choices and parameters that go into the development of an ALD process. Concepts such as the choice of precursors from the various deposition parameters that must be carefully controlled and tuned to produce optimal results. 

Chapter~\ref{chap:Charact} will introduce and briefly discuss the topics involved in characterizing the various materials to be utilized in the deposition process, as well as the various techniques used to analyze and quantify the properties of the produced samples. 

Subsequently, chapter~\ref{ch:Methods} will go into more detail of exactly how these analyses were performed, briefly mentioning various details of the standard measurement techniques as well as going into a bit more detail where analysis procedures were developed for application to this project. 

In chapter~\ref{ch:Results}, the details of the results and data collected from the various measurements and experiments that were performed during this study will be discussed thoroughly.  

Finally, chapter~\ref{ch:Conc} will draw conclusions from the entirety of the work done in the course of this project, as well as discuss possibilities for future experimentation and refinement of the process. A prime example of this would be to apply this research into the development of a procedure to produce films composed of the far more technically relevant PZT(52-48). 


%
%The primary scope of this project involved the creation of a method to produce lead titanate thin films. To this end, a number of different steps needed to be taken. Primarily, 