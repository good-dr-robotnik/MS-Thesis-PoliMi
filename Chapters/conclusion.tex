\chapter{Conclusions}
\label{ch:Conc}
\thispagestyle{empty}


The final result of this study was to show that it is certainly possible to grow perovskites using an ALD process, but the exact processes are sensitive and require much attention and further analysis.  

At the completion of this project, it can be said that the following goals have been met. A method for analyzing and predicting the behavior and operating windows of potential ALD precursors using standard thermal analysis techniques (TGA and DSC) has been proposed and shows good promise for applicability towards other material systems and precursor.\cite{Broido_TGA_1969,skoog_DSC_1998,wunderlich_thermal_1990}

A reliable method of analyzing film thicknesses and growth rates has been developed utilizing ellipsometry, which allows for rapid and non-destructive testing (and possibly incorporation as an \emph{in situ} testing mechanism to monitor film growth during deposition). In addition, the method described is capable of extracting additional information about film properties (e.g. band gap energies and other electronic characteristics) that could be applied given a higher degree of phase purity in the final crystallized film.\cite{Bruzzese_2010,schubert_infrared_2005,tompkins_spectroscopic_1999}

Use of XRF, in place of the more common EDS, allowed for the quantitative analysis of film composition even at ultra thin thicknesses. This rapid and non-destructive tool was of significant utility to this project. However, the ability to perform surface sensitive measurements would have made analysis of films grown on strontium titanate crystals more complete.\cite{Vincze_XRF_2005}

It was found that phase identification was possible using XRD with the thin films produced in this study. However, utilization of grazing incidence XRD --- a technique used to improve surface sensitivity of XRD --- would likely improve upon the results seen here, and would likely prove necessary if films of lower thicknesses were to be analyzed. \cite{giacovazzo_XRD_1992}

It proved to be non-trivial to produce films which, upon annealing, crystallized into phase-pure perovskite \PTO{}. Instead, the films often degrade into a mixture of PTO, pyrochlore forms (\ce{PbTi3O7} and \ce{Pb2Ti2O6}), various lead oxides (\ce{PbO_{x}}), and rutile tiania. At this time it is unclear exactly what causes this behavior. Additional research into the thermodynamics and crystallization behavior of these films would be of great value. 

%%%%%%%%%%%%%%%%%%%%%%%%%%%%%%%%%%%%%%%%%%%%%%%%%%%%%%%%%%
%%%%%%%%%%%%%%%%%%%%%%%%%%%%%%%%%%%%%%%%%%%%%%%%%%%%%%%%%%
%%%%%%%%%%%%%%%%%%%%%%%%%%%%%%%%%%%%%%%%%%%%%%%%%%%%%%%%%%

\section{Future Work}
\label{sec:Conc-Future}

While the work presented herein is noteworthy, and lays a framework for further investigation and refinement of ALD deposited perovskite oxides including the topic of discussion of this thesis, there is much left to be investigated in this line of research. Next steps would serve to further refine the process to improve the reliability of deposition, improving the phase purity and improve the degree of epitaxy of the grown film, or better conserve and deliver precursor (issues that plagued this project throughout its course). Being able to consistently produce films of constant quality would make this work applicable to a myriad of material systems. 

Additionally, there is one aspect of film characterization that had not been investigated thoroughly during the course of this project: the ferroelectric character of the films. Verification of ferroelectric behavior, even initially in isolated grains, would greatly improve the value of the method. Initial tests would likely require the use of microscopy techniques, e.g. piezoelectric force microscopy (PFM), to measure response of isolated grains of ferroelectric material; more standard ferroelectric measurements could likely be utilized with improved film quality and phase purity. 

Following this path of discussion quickly leads to the question of doping the films in order to improve properties. The obvious example here would be the development of an ALD process for \ce{PbZr_{0.52}Ti_{0.48}O3}. Lead titanate as a material is rather uncommonly used in applications, however as a simplified example of this family it was a prime candidate for the primary research. This project thus serves as a gateway to exploring the more valuable and technically relevant materials via ALD. 

Finally, it would be useful to extend this method to completely new material families. \ce{BaSrTiO3} is another commonly used ferroelectric material. Use of this material is gaining interest not only due to its intrinsic properties, but also due to its lack of lead content which makes it a far more environmentally conscious choice. \ce{BiFeO3} and corresponding doped materials, are another interesting perovskite. This set of materials presents multiferroic --- a combination of (anti)ferromagnetism and ferroelectricity --- behavior, making it a prime target for such novel technologies as spintronics. 






