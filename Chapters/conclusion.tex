\chapter{Conclusions}
\label{ch:Conc}
\thispagestyle{empty}






%%%%%%%%%%%%%%%%%%%%%%%%%%%%%%%%%%%%%%%%%%%%%%%%%%%%%%%%%%
%%%%%%%%%%%%%%%%%%%%%%%%%%%%%%%%%%%%%%%%%%%%%%%%%%%%%%%%%%
%%%%%%%%%%%%%%%%%%%%%%%%%%%%%%%%%%%%%%%%%%%%%%%%%%%%%%%%%%

\section{Future Work}
\label{sec:Conc-Future}

While the work presented herein is noteworthy, and lays a framework for further investigation and refinement of ALD deposited perovskite oxides including the topic of discussion of this thesis, there is much left to be investigated in this line of research. Next steps would serve to further refine the process to improve the reliability of deposition, improving the phase purity and improve the degree of epitaxy of the grown film, or better conserve and deliver precursor (issues that plagued this project throughout its course). Being able to consistently produce films of constant quality would make this work applicable to a myriad of systems. 

Additionally, there is one aspect of film characterization that had not been investigated thoroughly during the course of this project: the ferroelectric character of the films. Verification of ferroelectric behavior, even initially in isolated grains, would greatly improve the value of the method. Initial tests would likely require the use of microscopy techniques, e.g. piezoelectric force microscopy (PFM); more standard ferroelectric measurements could likely be utilized with improved film quality. 

Following this path of discussion quickly leads to the question of doping the films in order to improve properties. As an example, it is well known \reword{[citation needed]} that a solid solution of \ce{PbTiO3} and \ce{PbZrO3} greatly improves \reword{polarizability as well as the energy required to switch the film (???).} 

Finally, it would be useful to extend this method to completely novel material systems. \ce{BaSrTiO3} is another commonly used ferroelectric material (and valuable due to its lack of lead content), and \ce{BiFeO3} is another interesting perovskite as it presents multiferroic behavior as opposed to simply ferroelectricity. 






